\documentclass{article}
% PACKAGES %
\usepackage[english]{} % Sets the language
\usepackage[margin=2cm]{geometry} % Sets the margin size
\usepackage{fancyhdr} % Allows creation of headers
\usepackage{xcolor} % Allows the use of color in text
\usepackage{float} % Allows figures and tables to be floats
\usepackage{appendix}
\usepackage{amsmath} % Enhanced math package prepared by the American Mathematical Society
	\DeclareMathOperator{\sech}{sech} % Include sech
\usepackage{amssymb} % AMS symbols package
\usepackage{mathrsfs}% More math symbols
\usepackage{breqn} % Allows line breaking in math mode
\usepackage{cancel} % Allows math strikethroughs to show cancellations
\usepackage{bm} % Allows you to use \bm{} to make any symbol bold
\usepackage{bbold} % Allows more bold characters
\usepackage{verbatim} % Allows you to include code snippets
\usepackage{setspace} % Allows you to change the spacing between lines at different points in the document
\usepackage{parskip} % Allows you alter the spacing between paragraphs
\usepackage{multicol} % Allows text division into multiple columns
\usepackage{units} % Allows fractions to be expressed diagonally instead of vertically
\usepackage{booktabs,multirow,multirow} % Gives extra table functionality
\usepackage[final]{pdfpages} % Allows pdfs to be imported
\usepackage{hyperref} % Allows hyperlinks in the document
\usepackage{rotating} % Allows tables to be rotated
\usepackage{graphicx} % Enhanced package for including graphics/figures
	% Set path to figure image files
	\graphicspath{ {fig/} }
\usepackage{listings} % for including text files
	\lstset{basicstyle=\ttfamily\scriptsize,
        		  keywordstyle=\color{blue}\ttfamily,
        	  	  stringstyle=\color{red}\ttfamily,
          	  commentstyle=\color{gray}\ttfamily,
          	 }		
\newcommand{\tab}{\-\hspace{1cm}}

\newcommand{\p}{\partial}
\newcommand{\ppt}{\frac{\p}{\p t}}
\newcommand{\grad}{\vec{\nabla}}

\newcommand{\Xs}{\Sigma}
\newcommand{\xs}{\sigma}

\newcommand{\Oov}{\frac{1}{v}}

\newcommand{\pos}{\vec{r}}
\newcommand{\cur}{\vec{J}}
\newcommand{\Oh}{\hat{\Omega}}

\newcommand{\intfp}{\int_{4\pi}}
\newcommand{\intzi}{\int_0^{\infty}}


\newcommand{\rt}{(\pos,t)}
\newcommand{\rE}{(\pos,E)}
\newcommand{\rEO}{(\pos,E,\Oh)}
\newcommand{\rEt}{(\pos,E,t)}
\newcommand{\rEtprime}{(\pos,E',t)}
\newcommand{\rEOprime}{(\pos,E',\Oh')}
\newcommand{\rOt}{(\pos,\Oh,t)}
\newcommand{\rOtprime}{(\pos,\Oh',t)}
\newcommand{\rEOt}{(\pos,E,\Oh,t)}
\newcommand{\rEOtprime}{(\pos,E',\Oh',t)}
\newcommand{\EO}{(E,\Oh)}
\newcommand{\EOprime}{(E',\Oh')}
\newcommand{\EOt}{(E,\Oh,t)}



% Create a header w/ Name & Date
\pagestyle{fancy}
\rhead{\textbf{Mitch Negus} \; 10/20/2017}

\begin{document}
\thispagestyle{empty}

{\bf {\large {NE250 Homework {4} \hfill Mitch Negus\\
		\hspace*{\fill} 10/29/2017\\ }}}
		
		
		
%%%%%%%%%%%%%%%%%%%%%%%%%%%%%%%%%% PROBLEM 1 %%%%%%%%%%%%%%%%%%%%%%%%%%%%%%%%%%

\section*{Problem 1}

\subsection*{\textit{a.})}

We are attempting to prove
\begin{align*}
\mu\frac{\p \psi}{\p r} + \frac{(1-\mu^2)}{r}\frac{\p \psi}{\p \mu} 
		&= \frac{\mu}{r^2}\frac{\p(r^2\psi)}{\p r} + \frac{\p}{\p \mu}\left[\frac{(1-\mu^2)\psi}{r}\right] \\
		&= \frac{\mu}{r^2}\left[r^2\frac{\p \psi}{\p r} + \frac{\p(r^2)}{\p r}\psi\right] + \frac{1}{r}\left[(1-\mu^2)\frac{\p \psi}{\p \mu} + \frac{\p(1-\mu^2)}{\p \mu}\psi\right] \\
				&= \frac{\mu}{r^2}\left[r^2\frac{\p \psi}{\p r} + 2r\psi\right] + \frac{1}{r}\left[(1-\mu^2)\frac{\p \psi}{\p \mu} - 2\mu\psi\right] \\
				&= \mu\frac{\p \psi}{\p r} + \frac{2\mu}{r}\psi + \frac{(1-\mu^2)}{r}\frac{\p \psi}{\p \mu} - \frac{2\mu}{r}\psi \\
\mu\frac{\p \psi}{\p r} + \frac{(1-\mu^2)}{r}\frac{\p \psi}{\p \mu}	&= \mu\frac{\p \psi}{\p r} + \frac{(1-\mu^2)}{r}\frac{\p \psi}{\p \mu}\\
\end{align*}
\tab\tab\tab$\square$

\subsection*{\textit{b.})}

If we now integrate the streaming operator over $-1 \leq \mu \leq 1$, we have
$$ \int_{-1}^{1} d\mu \frac{\mu}{r^2}\frac{\p(r^2\psi)}{\p r} + \frac{\p}{\p \mu}\left[\frac{(1-\mu^2)\psi}{r}\right] $$
$$ \int_{-1}^{1} d\mu \frac{\mu}{r^2}\frac{\p(r^2\psi)}{\p r} + \int_{-1}^{1} d\mu  \frac{\p}{\p \mu}\left[\frac{(1-\mu^2)\psi}{r}\right] $$
$$ \frac{1}{r^2} \int_{-1}^{1} d\mu \, \mu \frac{\p(r^2\psi)}{\p r} + \left[ \frac{(1-\mu^2)\psi}{r}\right]_{-1}^{1} $$
$$ \frac{1}{r^2} \frac{\p }{\p r} \left( r^2 \int_{-1}^{1} d\mu \, \mu \, \psi \right) .$$
If we note that $\cur = \intfp d\Oh \, \Oh \, \psi = \int_{-1}^{1} d\mu \, \mu \, \psi$ then,
$$\boxed{ \frac{1}{r^2} \frac{\p }{\p r} \left( r^2 \cur \right) }.$$

\subsection*{\textit{c.})}

Integrating this expression for the streaming operator over a spherical shell $(r_1 \leq r \leq r_2)$, we have

$$ \int_{r_1}^{r_2} 4 \pi r^2 \frac{1}{r^2} \frac{\p }{\p r} \left( r^2 \cur \right) dr $$
$$ 4 \pi \int_{r_1}^{r_2} \frac{\p }{\p r} \left( r^2 \cur \right) dr $$
$$ 4 \pi \left[ r^2 \cur \right]_{r_1}^{r_2} $$
$$\boxed{ 4 \pi \left[ r_2^2 \cur(r_2) - r_1^2 \cur(r_1) \right] }.$$



%%%%%%%%%%%%%%%%%%%%%%%%%%%%%%%%%% PROBLEM 2 %%%%%%%%%%%%%%%%%%%%%%%%%%%%%%%%%%

\section*{Problem 2}

\subsection*{\textit{a.})}

The streaming operator can first be defined as 
$$ \Oh \cdot \nabla \equiv \frac{\p}{\p s} .$$
We can then expand this for cylindrical geometries as we did for spherical geometries. In addition to being dependent on position, $\pos = (\rho,\theta,z)$, we note that now the angular flux depends on both angular variables describing the velocity, $\xi$, the cosine of the polar angle, and $\omega$ the azimuthal angle. 
$$ \frac{\p \psi}{\p s} = \frac{\p \psi}{\p \rho}\frac{\p \rho}{\p s} + \frac{\p \psi}{\p \theta}\frac{\p \theta}{\p s} + \frac{\p \psi}{\p z}\frac{\p z}{\p s} + \frac{\p \psi}{\p \xi}\frac{\p \xi}{\p s} + \frac{\p \psi}{\p \omega}\frac{\p \omega}{\p s} .$$
In the one dimensional case, $\frac{\p \psi}{\p \theta} = 0$ and $\frac{\p \psi}{\p z} = 0$. Furthermore, because our geometry is cylindrical, $\frac{\p \xi}{\p s} = 0$ (the particle's direction with respect to $\hat{z}$ remains constant),  so our equation for $\frac{\p \psi}{\p s}$ simplifies to
$$ \frac{\p \psi}{\p s} = \frac{\p \psi}{\p \rho}\frac{\p \rho}{\p s} + \frac{\p \psi}{\p \omega}\frac{\p \omega}{\p s} .$$
By analyzing the geometry further, we can determine from figure 1-16 in Lewis \& Miller figure that
$$ \frac{\p \rho}{\p s} = \mu \quad \text{and} \quad \frac{\p \omega}{\p s} = -\frac{\eta}{\rho}.$$
Finally, our streaming operator becomes
$$ \frac{\p \psi}{\p s} = \mu \frac{\p \psi}{\p \rho} -\frac{\eta}{\rho}\frac{\p \psi}{\p \omega} .$$
With manipulation (and specifically noting that $\mu = \frac{\p \eta}{\p \omega}$), we can reach the streaming operator in conservation form.
\begin{align*}
\left[ \Oh \cdot \nabla \right] \psi	&=  \mu \frac{\p \psi}{\p \rho} -\frac{\eta}{\rho}\frac{\p \psi}{\p \omega}\\
										&=  \frac{\mu \rho}{\rho} \frac{\p \psi}{\p \rho} + \frac{\mu\psi}{\rho} -\frac{\eta}{\rho}\frac{\p \psi}{\p \omega} - \frac{\mu\psi}{\rho} \\
										&= \frac{\mu}{\rho} \left(\rho \frac{\p \psi}{\p \rho} + \psi \right) - \frac{1}{\rho} \left( \eta\frac{\p \psi}{\p \omega} + \mu\psi \right)\\
										&= \frac{\mu}{\rho} \left(\rho \frac{\p \psi}{\p \rho} + \frac{\p \rho}{\p \rho} \psi \right) - \frac{1}{\rho} \left( \eta\frac{\p \psi}{\p \omega} + \frac{\p \eta}{\p \omega} \psi \right)
\end{align*}
$$\boxed{ \left[ \Oh \cdot \nabla \right] \psi = \frac{\mu}{\rho} \frac{\p}{\p \rho}\left( \rho\psi \right) - \frac{1}{\rho} \frac{\p}{\p \omega}\left( \eta\psi \right) }.$$

\subsection*{\textit{b.})}

We can use this streaming operator in the transport equation to describe one dimensional cylindrical geometries. For simplicity, we will consider that the transport equation is describing a non-multiplying system. In one-dimensional cylindrical coordinate systems, dependence on position, $\pos$, now only depends on $\rho$, and velocity direction $\Oh$, depends on $\omega$ and $\xi$.
$$ \frac{1}{v}\frac{\p}{\p t}\psi(\rho,\omega,\xi,E,t) + \left[\frac{\mu}{\rho} \frac{\p}{\p \rho}\left( \rho\psi(\rho,\omega,\xi,E,t) \right) - \frac{1}{\rho} \frac{\p}{\p \omega}\left( \eta\psi(\rho,\omega,\xi,E,t) \right)\right] + \Xs_t(\rho,E) \psi(\rho,\omega,\xi,E,t) = s(\rho,\omega,\xi,E,t) $$
Integrating this equation over all angles ($0 \leq \omega \leq 2\pi$ and $-1 \leq \xi \leq 1$)
\begin{align*}
\int_0^{2\pi} d\omega \int_{-1}^{1} d\xi \, \frac{1}{v}\frac{\p}{\p t}\psi(\rho,\omega,\xi,E,t) + \int_0^{2\pi} d\omega \int_{-1}^{1} d\xi \, \left[ \frac{\mu}{\rho} \frac{\p}{\p \rho}\left( \rho\psi(\rho,\omega,\xi,E,t) \right) - \frac{1}{\rho} \frac{\p}{\p \omega}\left( \eta\psi(\rho,\omega,\xi,E,t) \right)\right] + \\
+ \int_0^{2\pi} d\omega \int_{-1}^{1} d\xi \,\Xs_t(\rho,E) \psi(\rho,\omega,\xi,E,t) = \int_0^{2\pi} d\omega \int_{-1}^{1} d\xi \, s(\rho,\omega,\xi,E,t)
\end{align*}
If we let $\int_0^{2\pi} d\omega \int_{-1}^{1} d\xi \, s(\rho,\omega,\xi,E,t) = S(\rho,E,t)$, then
$$ \frac{1}{v}\frac{\p}{\p t}\phi(\rho,E,t) + \int_0^{2\pi} d\omega \int_{-1}^{1} d\xi \, \left[ \frac{\mu}{\rho} \frac{\p}{\p \rho}\left( \rho\psi(\rho,\omega,\xi,E,t) \right) - \frac{1}{\rho} \frac{\p}{\p \omega}\left( \eta\psi(\rho,\omega,\xi,E,t) \right)\right] + \Xs_t(\rho,E) \phi(\rho,E,t) = S(\rho,E,t) $$
Again, consulting our geometry, we find $\mu = \sqrt{1-\xi^2}\cos\omega$ and $\eta = \sqrt{1-\xi^2}\sin\omega$.
\begin{align*}
\frac{1}{v}\frac{\p}{\p t}\phi(\rho,E,t) + \int_0^{2\pi} d\omega \int_{-1}^{1} d\xi \, \left[ \frac{\sqrt{1-\xi^2}\cos\omega}{\rho} \frac{\p}{\p \rho}\left( \rho\psi(\rho,\omega,\xi,E,t) \right) - \frac{1}{\rho} \frac{\p}{\p \omega}\left( \sqrt{1-\xi^2}\sin\omega \psi(\rho,\omega,\xi,E,t) \right)\right] + \\
+ \Xs_t(\rho,E) \phi(\rho,E,t) = S(\rho,E,t)
\end{align*}
\begin{align*}
\frac{1}{v}\frac{\p}{\p t}\phi(\rho,E,t) + \int_0^{2\pi} d\omega \int_{-1}^{1} d\xi \, \left[ \frac{\sqrt{1-\xi^2}\cos\omega}{\rho} \frac{\p}{\p \rho}\left( \rho\psi(\rho,\omega,\xi,E,t) \right)\right] + \\
- \int_0^{2\pi} d\omega \int_{-1}^{1} d\xi \, \left[ \frac{1}{\rho} \frac{\p}{\p \omega}\left( \sqrt{1-\xi^2}\sin\omega \psi(\rho,\omega,\xi,E,t) \right)\right] + \\
+ \Xs_t(\rho,E) \phi(\rho,E,t) = S(\rho,E,t)
\end{align*}
\begin{align*}
\frac{1}{v}\frac{\p}{\p t}\phi(\rho,E,t) + \frac{1}{\rho}\frac{\p}{\p \rho}\left(\rho \int_0^{2\pi} d\omega \int_{-1}^{1} d\xi \, \sqrt{1-\xi^2}\cos\omega \; \psi(\rho,\omega,\xi,E,t) \right) + \\
-\frac{1}{\rho} \int_{-1}^{1} d\xi \int_0^{2\pi} d\omega \, \frac{\p}{\p \omega}\left( \sqrt{1-\xi^2}\sin\omega \; \psi(\rho,\omega,\xi,E,t) \right) + \\
+ \Xs_t(\rho,E) \phi(\rho,E,t) = S(\rho,E,t)
\end{align*}
\begin{align*}
\frac{1}{v}\frac{\p}{\p t}\phi(\rho,E,t) + \frac{1}{\rho}\frac{\p}{\p \rho}\left(\rho \int_0^{2\pi} d\omega \cos\omega \int_{-1}^{1} d\xi \, \sqrt{1-\xi^2} \psi(\rho,\omega,\xi,E,t) \right) + \\
-\frac{1}{\rho} \int_{-1}^{1} d\xi \sqrt{1-\xi^2} \left[ \sin\omega \; \psi(\rho,\omega,\xi,E,t) \right]_0^{2\pi} + \\
+ \Xs_t(\rho,E) \phi(\rho,E,t) = S(\rho,E,t)
\end{align*}
$$ \frac{1}{v}\frac{\p}{\p t}\phi(\rho,E,t) + \frac{1}{\rho}\frac{\p}{\p \rho}\left(\rho \int_0^{2\pi} d\omega \cos\omega \int_{-1}^{1} d\xi \, \sqrt{1-\xi^2} \psi(\rho,\omega,\xi,E,t) \right) + \Xs_t(\rho,E) \phi(\rho,E,t) = S(\rho,E,t) $$
$$...$$



%%%%%%%%%%%%%%%%%%%%%%%%%%%%%%%%%% PROBLEM 3 %%%%%%%%%%%%%%%%%%%%%%%%%%%%%%%%%%

\section*{Problem 3}

The transport equation in one dimension is
$$ \mu \cdot \frac{\p}{\p x}\psi(x,\mu,E) + \Xs_t \psi(x,\mu,E) = q(x,\mu,E) $$
If we look along a characteristic ``curve'' in one dimension, we can say
$$ \frac{\p \psi}{\p s} = \mu \frac{\p \psi}{\p x} $$
and
$$ \frac{\p}{\p s} \psi(x_0+\mu s,\mu,E) + \Xs_t \psi(x_0+\mu s,\mu,E) = q(x_0+\mu s,\mu,E) $$
Since our derivation of the integral form of the transport equation (in class) did not explicitly require the fact that our problem was three-dimensional, we can use the same method for the one-dimensional case. We arrive at the following:
$$ \psi(x,\mu,E) = \intzi d\rho' \exp \left[ -\int_0^{\rho'} d\rho'' \, \Xs_t(x-\rho''\mu,E)\right] q(x-\rho'\mu,\mu,E) $$
where now the exponential represents the attenuation of neutrons as they move from $x-\rho' \mu$ to $x$, and the source term represents the production of neutrons at $x-\rho'\mu$ into $(\mu,E)$.

Since the cross section is uniform in the slab $\Xs_t(x-\rho''\mu,E) = \Xs_t$, and
$$ \psi(x,\mu,E) = \intzi d\rho' \exp \left[ -\int_0^{\rho'} d\rho'' \, \Xs_t\right] q(x-\rho'\mu,\mu,E) $$
$$ \psi(x,\mu,E) = \intzi d\rho' \exp \left[ -\Xs_t \int_0^{\rho'} d\rho''\right] q(x-\rho'\mu,\mu,E) $$
$$ \psi(x,\mu,E) = \intzi d\rho' \exp \left[ -\Xs_t \rho'\right] q(x-\rho'\mu,\mu,E) $$
The uncollided flux will consist entirely of neutrons with energy $E_0$ (if the neutron source beam is monoenergetic, neutrons must collide to have energy other than $E_0$), and direction $\mu = 1$ (if the neutrons are traveling in some other direction, they must have scattered into that direction)
$$ \psi_0(x,\mu,E) = \psi_0(x,1,E_0) = \intzi d\rho' \exp \left[ -\Xs_t \rho'\right] q(x-\rho',1,E_0) $$
Since we are ignoring fission, we note that the source term is simply the beam source, $q = s$.

The boundary conditions for this problem are
\begin{enumerate}
\item $$ J_+(0) = \int_0^1 \intzi \psi(0,\mu,E) d\mu dE = I $$
\item $$ J_-\left(\frac{a}{\tilde{\mu}}\right) = \int_{-1}^0 \intzi \psi\left(\frac{a}{\tilde{\mu}},\mu,E\right) d\mu \, dE = 0 $$
\end{enumerate}

Using our first boundary condition, we can treat the source $s$ as if it is a monodirectional point source appearing at the slab boundary, $x=0$. 
$$ \psi_0(x,\mu,E) = \psi_0(x,1,E_0) = \intzi d\rho' \exp \left[ -\Xs_t \rho'\right] s(x-\rho',1,E_0)\delta(x-\rho') $$
$$ \psi_0(x,\mu,E) = \psi_0(x,1,E_0) = I e^{-\Xs_t x} $$



%%%%%%%%%%%%%%%%%%%%%%%%%%%%%%%%%% PROBLEM 4 %%%%%%%%%%%%%%%%%%%%%%%%%%%%%%%%%%

\section*{Problem 4}




%%%%%%%%%%%%%%%%%%%%%%%%%%%%%%%%%% PROBLEM 5 %%%%%%%%%%%%%%%%%%%%%%%%%%%%%%%%%%

\section*{Problem 5}

The integral form of the transport equation is
$$ \psi(\pos,\Oh,E) = \intzi d\rho' \exp \left[ -\int_0^{\rho'} d\rho'' \, \Xs_t(\pos-\rho''\Oh,E)\right] q(\pos-\rho'\Oh,\Oh,E) $$
For isotropic sources (scattering and fission are not present in a purely absorbing medium), the source term becomes
$$ q(\pos-\rho'\Oh,\Oh,E) = \frac{S(\pos-\rho'\Oh,E)}{4\pi} $$
Where the source is a point source, 
$$ q(\pos-\rho'\Oh,\Oh,E) = \frac{S(E)}{4\pi}\delta(\pos-\rho'\Oh) $$
Using this in the integral form of the TE, and noting that $\Xs_t(E) = \Xs_t$,
$$ \psi(\pos,\Oh,E) = \intzi d\rho' \exp \left[ -\int_0^{\rho'} d\rho'' \, \Xs_t \right] \frac{S(E)}{4\pi}\delta(\pos-\rho'\Oh) $$
$$ \psi(\pos,\Oh,E) = \intzi d\rho' e^{-\Xs_t \rho'} \frac{S(E)}{4\pi}\delta(\pos-\rho'\Oh) $$
We can integrate this expression over angle to find the scalar flux.
$$ \phi(\pos,E) = \intfp d\Oh \intzi d\rho' e^{-\Xs_t \rho'} \frac{S(E)}{4\pi}\delta(\pos-\rho'\Oh) $$
When doing this integration, we can note that
$$ d\Oh = \frac{dA}{\rho'\,^2} $$
$$ d\rho' \, dA = dV $$
$$ \intfp d\Oh \intzi d\rho' = \int_V \frac{dV}{\rho'\, ^2} $$
and $\rho' = |\pos - \pos\,'|$, so that
$$ \phi(\pos,E) = \int_V \frac{dV}{\rho'\, ^2} e^{-\Xs_t \rho'} \frac{S(E)}{4\pi}\delta(\pos-\rho'\Oh) $$
$$ \phi(\pos,E) = \int_V \frac{d^3 \pos\,'}{4\pi |\pos-\pos\,'|^2} e^{-\Xs_t |\pos - \pos\,'|}S(E)\delta(\pos\,') $$
The delta function in this equation selects only the value for when $\pos\,' = 0$. 
$$ \phi(\pos,E) = \frac{1}{4\pi |\pos|^2} e^{-\Xs_t |\pos|}S(E) $$
Letting $r = |\pos|$,
$$\boxed{ \phi(r,E) = \frac{S(E)}{4\pi r^2} e^{-\Xs_t r} }.$$



%%%%%%%%%%%%%%%%%%%%%%%%%%%%%%%%%% PROBLEM 6 %%%%%%%%%%%%%%%%%%%%%%%%%%%%%%%%%%

\section*{Problem 6}




%%%%%%%%%%%%%%%%%%%%%%%%%%%%%%%%%% PROBLEM 7 %%%%%%%%%%%%%%%%%%%%%%%%%%%%%%%%%%

\section*{Problem 7}

For cylindrical geometries, the diffusion equation takes the form:
$$ \frac{1}{r}\frac{\p}{\p r}\left(r \frac{\p \phi(\pos)}{\p r}\right) + \frac{\p^2 \phi(\pos)}{\p z^2} + B^2\phi(\pos) = 0$$
where $B^2 = \frac{\nu\Sigma_f - \Sigma_a}{D} $ near criticality. 
Since our cylinder is infinite we lose dependence on $z$ in our equations.
$$ \frac{1}{r}\frac{\p}{\p r}\left(r \frac{\p \phi(r)}{\p r}\right) + B^2\phi(r) = 0$$
Solutions to this equation are are zeroth order Bessel functions of first and second kind, $J_0(r)$ and $Y_0(r)$. 
Then,
$$ \phi_b(r) = A_1 J_0(B r) + A_2 Y_0(B r) .$$
Our boundary conditions demand that the flux is finite at $r = 0$, and so $A_2 = 0$. 
Letting $A_1 = A$, we have
$$ \phi_b(r) = A J_0(B r) .$$
For a bare reactor, boundary conditions force the flux to go to zero at $\tilde{R}_b$,
$$ \phi(\tilde{R}_b) = 0 ,$$
and since the flux must be nonnegative, only the first zero of $J_0$ can satisfy this condition. This occurs at $B \tilde{R}_b = 2.4048$. $B = B_g = \frac{2.4048}{\tilde{R}_b}$. The criticality condition for the bare core is then 
$$ \left(\frac{\nu\Xs_f - \Xs_a}{D}\right)^2 = \left(\frac{2.4048}{\tilde{R}_b}\right)^2 $$
and the critical radius is
$$ \tilde{R}_b = \frac{2.4048 D}{\nu\Sigma_f - \Sigma_a} .$$

For a reflected core, our initial equation and the form of the solutions stay the same within the core region. 
With a finite flux at $r=0$, $A_2 = 0$ and if $A_1 = A$, then in the core 
$$ \phi_{r}(r) = A J_0(B r), \qquad r<R.$$
In the reflected region, we still have no $z$ dependence, but our diffusion equation is just
$$ \frac{1}{r}\frac{\p}{\p r}\left(r \frac{\p \phi(r)}{\p r}\right) - \frac{1}{L^2}\phi(r) = 0$$
without the dependence on fission. Solutions to this equation are zeroth order modified bessel functions of the first and second kind, $I_0(r)$ and $K_0(r)$.
In the reflector
$$ \phi_{r}(r) = C_1 I_0(\frac{r}{L}) + C_2 K_0(\frac{r}{L}), \qquad R<r<a$$
We can now impose the boundary condition that the flux must go to zero at the extrapolated distance in the reflector,
$$ \phi_{r}(R+\tilde{a}) = 0 .$$
We find
$$ 0 = C_1 I_0(\frac{R+\tilde{a}}{L}) + C_2 K_0(\frac{R+\tilde{a}}{L}) .$$
$$ C_2 = -C_1 \frac{I_0(\frac{R+\tilde{a}}{L})}{K_0(\frac{R+\tilde{a}}{L})} $$
If we let $C_1 = C$, then in the reflector
$$ \phi_{r}(r) = C I_0(\frac{r}{L}) - C \frac{I_0(\frac{R+\tilde{a}}{L})}{K_0(\frac{R+\tilde{a}}{L})} K_0(\frac{r}{L}), \qquad R<r<a$$
Finally, we use our boundary condition that the flux must be equal at the boundaries.
$$ \phi_{r}(R) = A J_0(B R) = C I_0(\frac{R}{L}) - C \frac{I_0(\frac{R+\tilde{a}}{L})}{K_0(\frac{R+\tilde{a}}{L})} K_0(\frac{R}{L}) $$

...

%%%%%%%%%%%%%%%%%%%%%%%%%%%%%%%%%% PROBLEM 8 %%%%%%%%%%%%%%%%%%%%%%%%%%%%%%%%%%

\section*{Problem 8}

The one-dimensional, single energy diffusion equation in each of the material regions are as follows:

\textbf{Multiplying Core:}
$$ \frac{d^2\phi(x)}{dx^2} + B^2\phi(x) = 0 , \qquad 0 < x < a $$
where $B^2 = \frac{\Xs_f - \Xs_a}{D}.$ 

\textbf{Uniform Source Reflector:}
$$ \frac{d^2\phi(x)}{dx^2} - \frac{1}{L^2}\phi(x) = \frac{-S_0}{D}, \qquad x > a $$
where $L^2 = \frac{D}{\Xs_a}$.

Boundary Conditions:
\begin{enumerate}
\item $ J_+(0) = 0 $
\item $ \phi_c(a) = \phi_r(a) $
\item $ \cur_c(a) = \cur_r(a) $
\item $ \phi(x) < \infty $
\end{enumerate}

In the multiplying core, solutions to the differential equation are
$$ \phi(x) = A_1\sin(Bx) + A_2\cos(Bx) $$
We can relate $J_+$ to the flux using the relation
$$ J_+(0) = 0 = \frac{\phi(0)}{4} - \frac{D}{2}\frac{d\phi}{dx}\bigg|_{x=0} $$
$$ 0 = \frac{A_2}{4} - \frac{D}{2}\left[A_1 B \cos(Bx) - A_2 B \sin(Bx)\right]_{x=0} $$
$$ 0 = \frac{A_2}{4} - \frac{A_1 B D}{2} $$
$$ A_2 = 2 A_1 B D $$
Letting $A_1 = A$,
$$ \phi_c(x) = A\sin(Bx) + 2ABD\cos(Bx) $$
In the uniform reflector, the homogeneous solution will be
$$ \phi_h(x) = C_1 e^{x/L} + C_2 e^{-x/L} ,$$
the particular solution will be
$$ \phi_p(x) = \frac{S_0L^2}{D} ,$$
and thus the general solution in the reflector will be
$$ \phi_r(x) = C_1 e^{x/L} + C_2 e^{-x/L} + \frac{S_0L^2}{D} $$
Noting that $\phi(x) < \infty$ for all $x$, we see that this is violated in the above equation as $x \rightarrow \infty$ if $C_1 \neq 0$. Letting $C_2 = C$, 
$$ \phi_r(x) = C e^{-x/L} + \frac{S_0L^2}{D} $$
Using our interface condition, we can see that 
$$ \phi_c(a) = \phi_r(a) $$
$$ A\sin(Ba) + 2ABD\cos(Ba) = C e^{-a/L} + \frac{S_0L^2}{D} $$ 
$$ C = AD \frac{\sin(Ba) + 2BD\cos(Ba) - S_0L^2}{De^{-a/L}} $$
In total, the flux in this system is
$$\boxed{ \phi(x) = \begin{cases} 	A \left( \sin(Bx) + 2 BD \cos(Bx) \right), & 0<x<a \\
							AD\left( \frac{\sin(Ba) + 2BD\cos(Ba) - S_0L^2}{De^{-a/L}}\right) e^{-x/L} + \frac{S_0L^2}{D}, & x > a \end{cases} }$$


%%%%%%%%%%%%%%%%%%%%%%%%%%%%%%%%%% PROBLEM 9 %%%%%%%%%%%%%%%%%%%%%%%%%%%%%%%%%%

\section*{Problem 9}




%\includepdf[pages=-]{NE250_HW04_mnegus-prob{...}.pdf}


\end{document}

For problem 2, used Applied Reactor Physics, by Alain Hébert






 
