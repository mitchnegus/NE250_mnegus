\documentclass{article}
% PACKAGES %
\usepackage[english]{} % Sets the language
\usepackage[margin=2cm]{geometry} % Sets the margin size
\usepackage{fancyhdr} % Allows creation of headers
\usepackage{xcolor} % Allows the use of color in text
\usepackage{float} % Allows figures and tables to be floats
\usepackage{appendix}
\usepackage{amsmath} % Enhanced math package prepared by the American Mathematical Society
	\DeclareMathOperator{\sech}{sech} % Include sech
\usepackage{amssymb} % AMS symbols package
\usepackage{mathrsfs}% More math symbols
\usepackage{breqn} % Allows line breaking in math mode
\usepackage{cancel} % Allows math strikethroughs to show cancellations
\usepackage{bm} % Allows you to use \bm{} to make any symbol bold
\usepackage{bbold} % Allows more bold characters
\usepackage{verbatim} % Allows you to include code snippets
\usepackage{setspace} % Allows you to change the spacing between lines at different points in the document
\usepackage{parskip} % Allows you alter the spacing between paragraphs
\usepackage{multicol} % Allows text division into multiple columns
\usepackage{units} % Allows fractions to be expressed diagonally instead of vertically
\usepackage{booktabs,multirow,multirow} % Gives extra table functionality
\usepackage[final]{pdfpages} % Allows pdfs to be imported
\usepackage{hyperref} % Allows hyperlinks in the document
\usepackage{rotating} % Allows tables to be rotated
\usepackage{graphicx} % Enhanced package for including graphics/figures
	% Set path to figure image files
	\graphicspath{ {fig/} }
\usepackage{listings} % for including text files
	\lstset{basicstyle=\ttfamily\scriptsize,
        		  keywordstyle=\color{blue}\ttfamily,
        	  	  stringstyle=\color{red}\ttfamily,
          	  commentstyle=\color{gray}\ttfamily,
          	 }		
\newcommand{\tab}{\-\hspace{1cm}}

\newcommand{\p}{\partial}
\newcommand{\ppt}{\frac{\p}{\p t}}
\newcommand{\grad}{\vec{\nabla}}

\newcommand{\Xs}{\Sigma}
\newcommand{\xs}{\sigma}

\newcommand{\Oov}{\frac{1}{v}}

\newcommand{\pos}{\vec{r}}
\newcommand{\cur}{\vec{J}}
\newcommand{\Oh}{\hat{\Omega}}

\newcommand{\intfp}{\int_{4\pi}}
\newcommand{\intzi}{\int_0^{\infty}}


\newcommand{\rt}{(\pos,t)}
\newcommand{\rE}{(\pos,E)}
\newcommand{\rEO}{(\pos,E,\Oh)}
\newcommand{\rEt}{(\pos,E,t)}
\newcommand{\rEtprime}{(\pos,E',t)}
\newcommand{\rEOprime}{(\pos,E',\Oh')}
\newcommand{\rOt}{(\pos,\Oh,t)}
\newcommand{\rOtprime}{(\pos,\Oh',t)}
\newcommand{\rEOt}{(\pos,E,\Oh,t)}
\newcommand{\rEOtprime}{(\pos,E',\Oh',t)}
\newcommand{\EO}{(E,\Oh)}
\newcommand{\EOprime}{(E',\Oh')}
\newcommand{\EOt}{(E,\Oh,t)}



% Create a header w/ Name & Date
\pagestyle{fancy}
\rhead{\textbf{Mitch Negus} \; 10/20/2017}

\begin{document}
\thispagestyle{empty}

{\bf {\large {NE250 Homework {4} \hfill Mitch Negus\\
		\hspace*{\fill} 10/29/2017\\ }}}
		
		
		
%%%%%%%%%%%%%%%%%%%%%%%%%%%%%%%%%% PROBLEM 1 %%%%%%%%%%%%%%%%%%%%%%%%%%%%%%%%%%

\section*{Problem 1}

\subsection*{\textit{a.})}

We are attempting to prove
\begin{align*}
\mu\frac{\p \psi}{\p r} + \frac{(1-\mu^2)}{r}\frac{\p \psi}{\p \mu} 
		&= \frac{\mu}{r^2}\frac{\p(r^2\psi)}{\p r} + \frac{\p}{\p \mu}\left[\frac{(1-\mu^2)\psi}{r}\right] \\
		&= \frac{\mu}{r^2}\left[r^2\frac{\p \psi}{\p r} + \frac{\p(r^2)}{\p r}\psi\right] + \frac{1}{r}\left[(1-\mu^2)\frac{\p \psi}{\p \mu} + \frac{\p(1-\mu^2)}{\p \mu}\psi\right] \\
				&= \frac{\mu}{r^2}\left[r^2\frac{\p \psi}{\p r} + 2r\psi\right] + \frac{1}{r}\left[(1-\mu^2)\frac{\p \psi}{\p \mu} - 2\mu\psi\right] \\
				&= \mu\frac{\p \psi}{\p r} + \frac{2\mu}{r}\psi + \frac{(1-\mu^2)}{r}\frac{\p \psi}{\p \mu} - \frac{2\mu}{r}\psi \\
\mu\frac{\p \psi}{\p r} + \frac{(1-\mu^2)}{r}\frac{\p \psi}{\p \mu}	&= \mu\frac{\p \psi}{\p r} + \frac{(1-\mu^2)}{r}\frac{\p \psi}{\p \mu}\\
\end{align*}
\tab\tab\tab$\square$

\subsection*{\textit{b.})}

If we now integrate the streaming operator over $-1 \leq \mu \leq 1$, we have
$$ \int_{-1}^{1} d\mu \frac{\mu}{r^2}\frac{\p(r^2\psi)}{\p r} + \frac{\p}{\p \mu}\left[\frac{(1-\mu^2)\psi}{r}\right] $$
$$ \int_{-1}^{1} d\mu \frac{\mu}{r^2}\frac{\p(r^2\psi)}{\p r} + \int_{-1}^{1} d\mu  \frac{\p}{\p \mu}\left[\frac{(1-\mu^2)\psi}{r}\right] $$
$$ \frac{1}{r^2} \int_{-1}^{1} d\mu \, \mu \frac{\p(r^2\psi)}{\p r} + \left[ \frac{(1-\mu^2)\psi}{r}\right]_{-1}^{1} $$
$$ \frac{1}{r^2} \frac{\p }{\p r} \left( r^2 \int_{-1}^{1} d\mu \, \mu \, \psi \right) .$$
If we note that $\cur = \intfp d\Oh \, \Oh \, \psi = \int_{-1}^{1} d\mu \, \mu \, \psi$ then,
$$\boxed{ \frac{1}{r^2} \frac{\p }{\p r} \left( r^2 \cur \right) }.$$

\subsection*{\textit{c.})}

Integrating this expression for the streaming operator over a spherical shell $(r_1 \leq r \leq r_2)$, we have

$$ \int_{r_1}^{r_2} 4 \pi r^2 \frac{1}{r^2} \frac{\p }{\p r} \left( r^2 \cur \right) dr $$
$$ 4 \pi \int_{r_1}^{r_2} \frac{\p }{\p r} \left( r^2 \cur \right) dr $$
$$ 4 \pi \left[ r^2 \cur \right]_{r_1}^{r_2} $$
$$\boxed{ 4 \pi \left[ r_2^2 \cur(r_2) - r_1^2 \cur(r_1) \right] }.$$



%%%%%%%%%%%%%%%%%%%%%%%%%%%%%%%%%% PROBLEM 2 %%%%%%%%%%%%%%%%%%%%%%%%%%%%%%%%%%

\section*{Problem 2}

The streaming operator can first be defined as 
$$ \Oh \cdot \nabla \equiv \frac{d}{ds} .$$
We can then expand this for cylindrical geometries as we did for spherical geometries. In addition to being dependent on position, $\pos = (\rho,\theta,z)$, we note that now the angular flux depends on both angular variables describing the velocity, $\xi$, the cosine of the polar angle, and $\omega$ the azimuthal angle. 
$$ \frac{d\psi}{ds} = \frac{d\psi}{d\rho}\frac{d\rho}{ds} + \frac{d\psi}{d\theta}\frac{d\theta}{ds} + \frac{d\psi}{dz}\frac{dz}{ds} + \frac{d\psi}{d\xi}\frac{d\xi}{ds} + \frac{d\psi}{d\omega}\frac{d\omega}{ds} .$$
In the one dimensional case, $\frac{d\psi}{d\theta} = 0$ and $\frac{d\psi}{dz} = 0$. Furthermore, because our geometry is cylindrical, $\frac{d\xi}{ds} = 0$ (the particle's direction with respect to $\hat{z}$ remains constant),  so our equation for $\frac{d\psi}{ds}$ simplifies to
$$ \frac{d\psi}{ds} = \frac{d\psi}{d\rho}\frac{d\rho}{ds} + \frac{d\psi}{d\omega}\frac{d\omega}{ds} .$$
By analyzing the geometry further, we can determine from figure 1-16 in Lewis \& Miller figure that
$$ \frac{d\rho}{ds} = \mu \quad \text{and} \quad \frac{d\omega}{ds} = -\frac{\eta}{\rho}.$$
Finally, our streaming operator becomes
$$ \frac{d\psi}{ds} = \mu \frac{d\psi}{d\rho} -\frac{\eta}{\rho}\frac{d\psi}{d\omega} .$$
With manipulation (and specifically noting that $\mu = \frac{\p \eta}{\p \omega}$), we can reach the streaming operator in conservation form.
\begin{align*}
\left[ \Oh \cdot \nabla \right] \psi	&=  \mu \frac{\p \psi}{\p \rho} -\frac{\eta}{\rho}\frac{\p \psi}{\p \omega}\\
										&=  \frac{\mu \rho}{\rho} \frac{\p \psi}{\p \rho} + \frac{\mu\psi}{\rho} -\frac{\eta}{\rho}\frac{\p \psi}{\p \omega} - \frac{\mu\psi}{\rho} \\
										&= \frac{\mu}{\rho} \left(\rho \frac{\p \psi}{\p \rho} + \psi \right) - \frac{1}{\rho} \left( \eta\frac{\p \psi}{\p \omega} + \mu\psi \right)\\
										&= \frac{\mu}{\rho} \left(\rho \frac{\p \psi}{\p \rho} + \frac{\p \rho}{\p \rho} \psi \right) - \frac{1}{\rho} \left( \eta\frac{\p \psi}{\p \omega} + \frac{\p \eta}{\p \omega} \psi \right)
\end{align*}
$$\boxed{ \left[ \Oh \cdot \nabla \right] \psi = \frac{\mu}{\rho} \frac{\p}{\p \rho}\left( \rho\psi \right) - \frac{1}{\rho} \frac{\p}{\p \omega}\left( \eta\psi \right) }.$$



%%%%%%%%%%%%%%%%%%%%%%%%%%%%%%%%%% PROBLEM 3 %%%%%%%%%%%%%%%%%%%%%%%%%%%%%%%%%%

\section*{Problem 3}




%%%%%%%%%%%%%%%%%%%%%%%%%%%%%%%%%% PROBLEM 4 %%%%%%%%%%%%%%%%%%%%%%%%%%%%%%%%%%

\section*{Problem 4}




%%%%%%%%%%%%%%%%%%%%%%%%%%%%%%%%%% PROBLEM 5 %%%%%%%%%%%%%%%%%%%%%%%%%%%%%%%%%%

\section*{Problem 5}




%%%%%%%%%%%%%%%%%%%%%%%%%%%%%%%%%% PROBLEM 6 %%%%%%%%%%%%%%%%%%%%%%%%%%%%%%%%%%

\section*{Problem 6}




%%%%%%%%%%%%%%%%%%%%%%%%%%%%%%%%%% PROBLEM 7 %%%%%%%%%%%%%%%%%%%%%%%%%%%%%%%%%%

\section*{Problem 7}

For cylindrical geometries, the diffusion equation takes the form:
$$ \frac{1}{r}\frac{\p}{\p r}\left(r \frac{\p \phi(\pos)}{\p r}\right) + \frac{\p^2 \phi(\pos)}{\p z^2} + B^2\phi(\pos) = 0$$
where $B^2 = \frac{\nu\Sigma_f - \Sigma_a}{D} $ near criticality. 
Since our cylinder is infinite we lose dependence on $z$ in our equations.
$$ \frac{1}{r}\frac{\p}{\p r}\left(r \frac{\p \phi(r)}{\p r}\right) + B^2\phi(r) = 0$$
Solutions to this equation are are zeroth order Bessel functions of first and second kind, $J_0(r)$ and $Y_0(r)$. 
Then,
$$ \phi_b(r) = A_1 J_0(B r) + A_2 Y_0(B r) .$$
Our boundary conditions demand that the flux is finite at $r = 0$, and so $A_2 = 0$. 
Letting $A_1 = A$, we have
$$ \phi_b(r) = A J_0(B r) .$$
For a bare reactor, boundary conditions force the flux to go to zero at $\tilde{R}_b$,
$$ \phi(\tilde{R}_b) = 0 ,$$
and since the flux must be nonnegative, only the first zero of $J_0$ can satisfy this condition. This occurs at $B \tilde{R}_b = 2.4048$. $B = B_g = \frac{2.4048}{\tilde{R}_b}$. The criticality condition for the bare core is then 
$$ \left(\frac{\nu\Xs_f - \Xs_a}{D}\right)^2 = \left(\frac{2.4048}{\tilde{R}_b}\right)^2 $$
and the critical radius is
$$ \tilde{R}_b = \frac{2.4048 D}{\nu\Sigma_f - \Sigma_a} .$$

For a reflected core, our initial equation and the form of the solutions stay the same within the core region. 
With a finite flux at $r=0$, $A_2 = 0$ and if $A_1 = A$, then in the core 
$$ \phi_{r}(r) = A J_0(B r), \qquad r<R.$$
In the reflected region, we still have no $z$ dependence, but our diffusion equation is just
$$ \frac{1}{r}\frac{\p}{\p r}\left(r \frac{\p \phi(r)}{\p r}\right) - \frac{1}{L^2}\phi(r) = 0$$
without the dependence on fission. Solutions to this equation are zeroth order modified bessel functions of the first and second kind, $I_0(r)$ and $K_0(r)$.
In the reflector
$$ \phi_{r}(r) = C_1 I_0(\frac{r}{L}) + C_2 K_0(\frac{r}{L}), \qquad R<r<a$$
We can now impose the boundary condition that the flux must go to zero at the extrapolated distance in the reflector,
$$ \phi_{r}(R+\tilde{a}) = 0 .$$
We find
$$ 0 = C_1 I_0(\frac{R+\tilde{a}}{L}) + C_2 K_0(\frac{R+\tilde{a}}{L}) .$$
$$ C_2 = -C_1 \frac{I_0(\frac{R+\tilde{a}}{L})}{K_0(\frac{R+\tilde{a}}{L})} $$
If we let $C_1 = C$, then in the reflector
$$ \phi_{r}(r) = C I_0(\frac{r}{L}) - C \frac{I_0(\frac{R+\tilde{a}}{L})}{K_0(\frac{R+\tilde{a}}{L})} K_0(\frac{r}{L}), \qquad R<r<a$$
Finally, we use our boundary condition that the flux must be equal at the boundaries.
$$ \phi_{r}(R) = A J_0(B R) = C I_0(\frac{R}{L}) - C \frac{I_0(\frac{R+\tilde{a}}{L})}{K_0(\frac{R+\tilde{a}}{L})} K_0(\frac{R}{L}) $$

...

%%%%%%%%%%%%%%%%%%%%%%%%%%%%%%%%%% PROBLEM 8 %%%%%%%%%%%%%%%%%%%%%%%%%%%%%%%%%%

\section*{Problem 8}

The one-dimensional, single energy diffusion equation in each of the material regions are as follows:

\textbf{Multiplying Core:}
$$ \frac{d\phi(x)}{dx} - B^2\phi(x) = 0 $$
where $B^2 = \frac{\Xs_f - \Xs_a}{D}.$ 

\textbf{Uniform Source Reflector:}
\textbf{Multiplying Core:}
$$ \frac{d\phi(x)}{dx} - B^2\phi(x) = 0 $$
where $B^2 = \frac{\Xs_f - \Xs_a}{D}.$ 

%%%%%%%%%%%%%%%%%%%%%%%%%%%%%%%%%% PROBLEM 9 %%%%%%%%%%%%%%%%%%%%%%%%%%%%%%%%%%

\section*{Problem 9}




%\includepdf[pages=-]{NE250_HW04_mnegus-prob{...}.pdf}


\end{document}

For problem 2, used Applied Reactor Physics, by Alain Hébert






 
