\documentclass{article}
% PACKAGES %
\usepackage[english]{} % Sets the language
\usepackage[margin=2cm]{geometry} % Sets the margin size
\usepackage{fancyhdr} % Allows creation of headers
\usepackage{xcolor} % Allows the use of color in text
\usepackage{float} % Allows figures and tables to be floats
\usepackage{appendix}
\usepackage{amsmath} % Enhanced math package prepared by the American Mathematical Society
	\DeclareMathOperator{\sech}{sech} % Include sech
\usepackage{amssymb} % AMS symbols package
\usepackage{mathrsfs}% More math symbols
\usepackage{breqn} % Allows line breaking in math mode
\usepackage{cancel} % Allows math strikethroughs to show cancellations
\usepackage{bm} % Allows you to use \bm{} to make any symbol bold
\usepackage{bbold} % Allows more bold characters
\usepackage{verbatim} % Allows you to include code snippets
\usepackage{setspace} % Allows you to change the spacing between lines at different points in the document
\usepackage{parskip} % Allows you alter the spacing between paragraphs
\usepackage{multicol} % Allows text division into multiple columns
\usepackage{units} % Allows fractions to be expressed diagonally instead of vertically
\usepackage{booktabs,multirow,multirow} % Gives extra table functionality
\usepackage[final]{pdfpages} % Allows pdfs to be imported
\usepackage{hyperref} % Allows hyperlinks in the document
\usepackage{rotating} % Allows tables to be rotated
\usepackage{graphicx} % Enhanced package for including graphics/figures
	% Set path to figure image files
	\graphicspath{ {fig/} }
\usepackage{listings} % for including text files
	\lstset{basicstyle=\ttfamily\scriptsize,
        		  keywordstyle=\color{blue}\ttfamily,
        	  	  stringstyle=\color{red}\ttfamily,
          	  commentstyle=\color{gray}\ttfamily,
          	 }		
\newcommand{\tab}{\-\hspace{1cm}}

\newcommand{\p}{\partial}
\newcommand{\ppt}{\frac{\p}{\p t}}
\newcommand{\grad}{\vec{\nabla}}

\newcommand{\Xs}{\Sigma}
\newcommand{\xs}{\sigma}

\newcommand{\Oov}{\frac{1}{v}}

\newcommand{\pos}{\vec{r}}
\newcommand{\cur}{\vec{J}}
\newcommand{\Oh}{\hat{\Omega}}

\newcommand{\intfp}{\int_{4\pi}}
\newcommand{\intzi}{\int_0^{\infty}}


\newcommand{\rt}{(\pos,t)}
\newcommand{\rE}{(\pos,E)}
\newcommand{\rEO}{(\pos,E,\Oh)}
\newcommand{\rEt}{(\pos,E,t)}
\newcommand{\rEtprime}{(\pos,E',t)}
\newcommand{\rEOprime}{(\pos,E',\Oh')}
\newcommand{\rOt}{(\pos,\Oh,t)}
\newcommand{\rOtprime}{(\pos,\Oh',t)}
\newcommand{\rEOt}{(\pos,E,\Oh,t)}
\newcommand{\rEOtprime}{(\pos,E',\Oh',t)}
\newcommand{\EO}{(E,\Oh)}
\newcommand{\EOprime}{(E',\Oh')}
\newcommand{\EOt}{(E,\Oh,t)}



% Create a header w/ Name & Date
\pagestyle{fancy}
\rhead{\textbf{Mitch Negus} \; 12/1/2017}

\begin{document}
\thispagestyle{empty}

{\bf {\large {NE250 Homework {6} \hfill Mitch Negus\\
		\hspace*{\fill} 12/1/2017\\ }}}
		
		
		
%%%%%%%%%%%%%%%%%%%%%%%%%%%%%%%%%% PROBLEM 1 %%%%%%%%%%%%%%%%%%%%%%%%%%%%%%%%%%

\section*{Problem 1}

\subsection*{\textit{a.})}
Expected Value/Mean: 
\begin{align*}
\mu &= \int_{-\infty}^{\infty} x f(x) \, dx \\
	&= \int_0^a \frac{x}{a} \, dx  \\
	&= \frac{1}{2a}\left[x^2\right]_0^a
\end{align*}
$$\boxed{ \mu = \frac{a}{2} }$$
Variance:
\begin{align*}
\sigma^2	&= E[(X-\mu)^2] \\
			&= \int_{-\infty}^{\infty}(x-\mu)^2 f(x) \, dx \\
			&= \int_0^a \left(x-\frac{a}{2}\right)^2 \left(\frac{1}{a}\right) \, dx \\
			&= \int_0^a \left(x^2-ax+\frac{a^2}{4}\right) \left(\frac{1}{a}\right) \, dx \\
			&= \int_0^a \frac{x^2}{a} \, dx - \int_0^a x \, dx + \int_0^a \frac{a}{4} \, dx \\
			&= \left[ \frac{x^3}{3a} \right]_0^a - \left[ \frac{x^2}{2} \right]_0^a + \left[ \frac{ax}{4} \right]_0^a \\
			&= \frac{a^2}{3} - \frac{a^2}{2} + \frac{a^2}{4}
\end{align*}
$$\boxed{ \sigma^2 = \frac{a^2}{12} }$$
Cumulative Distribution Function:
\begin{align*}
F(x)	&= \int_{-\infty}^{x} f(x') dx' \\
		&= \int_0^x \frac{1}{a} \, dx' \\
		&= \frac{1}{a}\left[x'\right]_0^x \\
		&= \frac{x}{a}
\end{align*}
$$\boxed{ F(x) = \begin{cases}	0, & x < 0, x > a \\
								\frac{x}{a}, & 0 < x < a \end{cases}}$$
\subsection*{\textit{b.})}
Expected Value/Mean: 
\begin{align*}
\mu &= \int_{-\infty}^{\infty} x f(x) \, dx \\
	&= \int_0^{\infty} \lambda x e^{-\lambda x} \, dx  \\
(\text{let }u = \lambda x, du &= \lambda\,dx, dv = e^{-\lambda x}, v = -e^{-\lambda x}/\lambda)\\
	&= uv|_0^{\infty} - \int_0^{\infty}v \, du \\
	&= -x e^{-\lambda x}\bigg|_0^{\infty} - \int_0^{\infty} \frac{-\lambda e^{-\lambda x}}{\lambda} \, dx \\
	&= 0 + \left[ \frac{-e^{-\lambda x}}{\lambda} \right]_0^{\infty}
\end{align*}
$$\boxed{ \mu = \frac{1}{\lambda} }$$
Variance:
\begin{align*}
\sigma^2	&= E[(X-\mu)^2] \\
			&= E[X^2] - \mu^2 \\
			&= \int_0^{\infty}x^2 f(x) \, dx - \frac{1}{\lambda^2}  \\
			&= \int_0^{\infty}\lambda x^2 e^{-\lambda x} \, dx - \frac{1}{\lambda^2}  \\
(\text{let }u = \lambda x^2, du &= 2\lambda x\,dx, dv = e^{-\lambda x}, v = -e^{-\lambda x}/\lambda)\\
			&= uv|_0^{\infty} - \int_0^{\infty}v \, du - \frac{1}{\lambda^2} \\
			&= \left[ -x^2 e^{-\lambda x}) \right]_0^{\infty} + \int_0^{\infty} 2 x e^{-\lambda x} \, dx - \frac{1}{\lambda^2}\\
			&= 0 + 2 \int_0^{\infty} x e^{-\lambda x} \, dx - \frac{1}{\lambda^2}\\
			&= \frac{2}{\lambda^2} \left[ e^{-\lambda x}(-\lambda x-1) \right]_0^{\infty} - \frac{1}{\lambda^2}\\
			&= \frac{2}{\lambda^2} - \frac{1}{\lambda^2}
\end{align*}
$$\boxed{ \sigma^2 = \frac{1}{\lambda^2} }$$
Cumulative Distribution Function:
\begin{align*}
F(x)	&= \int_{-\infty}^{x} f(x') dx' \\
		&= \int_0^x \lambda e^{-\lambda x'} \, dx' \\
		&= \left[-e^{-\lambda x'}\right]_0^x \\
		&= \left[1 -e^{-\lambda x} \right]
\end{align*}
$$\boxed{ F(x) = 1 -e^{-\lambda x} }$$



%%%%%%%%%%%%%%%%%%%%%%%%%%%%%%%%%% PROBLEM 2 %%%%%%%%%%%%%%%%%%%%%%%%%%%%%%%%%%

\section*{Problem 2}

(See Jupyter notebook, attached)



%%%%%%%%%%%%%%%%%%%%%%%%%%%%%%%%%% PROBLEM 3 %%%%%%%%%%%%%%%%%%%%%%%%%%%%%%%%%%

\section*{Problem 3}

\subsection*{\textit{a.})} 
\textit{True}. A normalized PDF is required so that the inverted CDF, from which we draw our sample, has a domain of 0 to 1. Then, selecting random numbers in this same interval will give us values of the PDF/CDF's independent variable distributed like the original PDF.

\subsection*{\textit{b.})}
\begin{multicols}{2}
\textbf{Deterministic Strengths} \\
\begin{enumerate}
\item Fast execution (all steps contribute to a meaningful solution)
\item No statistical uncertainties, solutions are exact (within truncation limits)
\item Solutions are global over the problem space
\end{enumerate}

\textbf{Monte Carlo Weaknesses} \\
\begin{enumerate}
\item Slow/inefficient (calculations for particles that do not contribute to the tally are essentially wasted)
\item Results contain statistical uncertainties
\item Solutions are only local (wherever tallies are collected)
\end{enumerate}

\columnbreak

\textbf{Monte Carlo Strengths} \\
\begin{enumerate}
\item Continuous (in space, energy, angle, etc.)
\item Easy to parallelize
\item No truncation error, solutions are exact (within uncertainties)
\end{enumerate}

\textbf{Deterministic Weaknesses} \\
\begin{enumerate}
\item Difficult to parallelize
\item Discretization (energy, angle, space, etc.) limits accuracy
\item Solutions contain truncation error (expansions are truncated)
\end{enumerate}

\end{multicols}

\subsection*{\textit{c.})}
The relative error of an analog Monte Carlo algorithm is $R = S_{\bar{x}}/{\bar{x}}$. Since $S_{\bar{x}} = S/\sqrt{N}$, it becomes apparent that $R \propto 1/\sqrt{N}$.

\subsection*{\textit{d.})}
The central limit theorem allows us to state, \underline{specifically for independently and identically distributed random variables}, that for many samples of $N$ measurements, the sample means will be distributed according to a normal distribution, with variance $\sigma^2/N$ (where $\sigma^2$ is the true variance of the original variable distribution).

\subsection*{\textit{e.})}
In a Monte Carlo algorithm, the tracking of particles through a geometry region requires the particle's mean free path in that region. At a boundary, the mean free path must be updated for proper transport through the new material. 

\subsection*{\textit{f.})}
The collision estimator is more likely to be accurate in cases where there are many collisions (such as a large and/or dense geometry region), and the majority of particles passing through the region of interest collide there. In the case that the geometry is optically thin, there will be few collisions in the region, and so a collision estimator will yield poor statistics. Then, a track length estimator will be preferable.

\subsection*{\textit{g.})}
The Consistent Adjoint Driven Importance Sampling method is termed "consistent" because it applies the appropriate weighting (to the nominal weight) for every particle that is born, according to the importance map defined by the adjoint source.



%%%%%%%%%%%%%%%%%%%%%%%%%%%%%%%%%% PROBLEM 4 %%%%%%%%%%%%%%%%%%%%%%%%%%%%%%%%%%

\section*{Problem 4}

(See Jupyter notebook, attached)




\includepdf[pages=-]{NE250_HW06_mnegus-prob2.pdf}
\includepdf[pages=-]{NE250_HW06_mnegus-prob4.pdf}


\end{document}





 
