\documentclass{report}
% PACKAGES %
\usepackage[english]{} % Sets the language
\usepackage[margin=2cm]{geometry} % Sets the margin size
\usepackage{fancyhdr} % Allows creation of headers
\usepackage{xcolor} % Allows the use of color in text
\usepackage{float} % Allows figures and tables to be floats
\usepackage{appendix}
\usepackage{amsmath} % Enhanced math package prepared by the American Mathematical Society
	\DeclareMathOperator{\sech}{sech} % Include sech
\usepackage{amssymb} % AMS symbols package
\usepackage{mathrsfs}% More math symbols
\usepackage{bm} % Allows you to use \bm{} to make any symbol bold
\usepackage{bbold} % Allows more bold characters
\usepackage{verbatim} % Allows you to include code snippets
\usepackage{setspace} % Allows you to change the spacing between lines at different points in the document
\usepackage{parskip} % Allows you alter the spacing between paragraphs
\usepackage{multicol} % Allows text division into multiple columns
\usepackage{units} % Allows fractions to be expressed diagonally instead of vertically
\usepackage{booktabs,multirow,multirow} % Gives extra table functionality
\usepackage{hyperref} % Allows hyperlinks in the document
\usepackage{rotating} % Allows tables to be rotated
\usepackage{graphicx} % Enhanced package for including graphics/figures
	% Set path to figure image files
	\graphicspath{ {} }
\usepackage{listings} % for including text files
	\lstset{basicstyle=\ttfamily\scriptsize,
        		  keywordstyle=\color{blue}\ttfamily,
        	  	  stringstyle=\color{red}\ttfamily,
          	  commentstyle=\color{gray}\ttfamily,
          	 }		
\newcommand{\tab}{\-\hspace{1cm}}

% Create a header w/ Name & Date
\pagestyle{fancy}
\rhead{\textbf{Mitch Negus} \; 9/5/2017}

\begin{document}
\thispagestyle{empty}

{\bf {\large {NE250 Homework {0} \hfill Mitch Negus\\
		\hspace*{\fill} 9/5/2017\\ }}}
\section*{RTGs}


- First public demonstration of RTG to pres Eisenhower, 1959 (1)\\
- Largely US produced (Russia has favored reactor based power systems) (3)\\
- Two components thermoelectric converter and heat source (1,2)\\
- Seebeck effect --> produces electricity from heat (1)\\
- Thermoelectric converter (1)\\
	-  temperature difference/gradient across heat source (1,2)\\
	-  conductor on hot end, p/n type semiconductors, load on cold end (still several hundred K) (1)\\
	- load (thermocouples) often connected in series to boost voltage output\\
	- relatively low conversion efficiency ~5-10\% (2,3)\\
- Heat source is a radioactive isotope, casing dispells heat into environment (1)\\
- Ideal for space exploration; high energy density, long lifetime d1,2)\\
- low power output is a challenge\\
- other considerations include high melting point, safe and cost effective to produce (2)\\
- mass considerations due to half-life; looking for reasonably equal power distribution over space mission \\(half-life ~ mission length) (2)\\
- goal is to keep penetrating emissions low (esp. for manned missions, also to prevent rad damage except locally, or in the event of catastrophic failure) (2)\\
- two manufacture methods: Used fuel isolation, neutron activation of target\\

-Dynamic systems being developed (better efficiencies ~25+\%) (2)\\

-even more benefit if radioactive decay leads to daughter decays (3)\\
-cerium 144 initially tested (plentiful from defense waste, 290 day half-life suited to military purposes) (2) - SNAP-1\\
-po210 used in SNAP-3 (demonstrated at WH in 1959-> to Ike) (2); half-life 138 days, still too short for many space missions (2)\\
-Sr90 used terrestrially, no space missions (requires heavy shielding) (2)\\
-others include H3 (12y), Am241(433y), Cm242(163d), Cm244(18y) (3)\\
-alpha emitting Am241 is esp. promising; produced by decay of Pu241 from snf\\

- pu238 is primary isotope used in RTGs (easy to produce stably, easy to encapsulate, emits absorbable radiation) (1)\\
- half-life of 87.7 yrs; low radiation, useful forms (2)\\
- domestic production ceased in 1986; availabilty has since been limited (purchasing from Russia) (2)\\
-1961 Transit 4A Navy nav sat = first nuclear power in space (2,3)\\
- as of 1991, PbTe and SiGe were primary RTG semiconductor components (1)\\
- SiGe beneficial for space, can withstand temps to 1000C (sublimation/oxidation effects suppressed by inert gas, vented once in space)\\
- Notable uses SNAP (systems for nuclear auxilary power) program (Apollo > 12, Pioneer, Viking), Voyager missions (1)\\
- SNAP 27 designed to be loaded modularly, plutonium microspheres (1)\\
- GPHS (gen purpose heat source used for Space shuttle ) (1)\\
- Used for all NASA missions after 1989 (2)\\
- GPHS module designed to deliver 250 W\_t (2); mass = 1.43 kg (2)\\

As nuclear batteries (4)\\
- RTGs inefficient and large due to thermoelectric conversion\\
- for batteries in general, scale lengths of system ought to be matched --- transport scale length of radiation, phyiscal dimension of energy conversion volume (not applicable to RTGs)\\
- electron--hole recombination for alpha/betavoltaics\\
- solid state recombination\\

Radioactive Heat Sources (RHS) (5)\\
- used in radioactive heater units (RHUs) or RTGs (5)\\


(1) DM Rowe
(2) DOE Germantown
(3) Europe
(4) Nuclear batteries
(5) modern RHS

\end{document}




