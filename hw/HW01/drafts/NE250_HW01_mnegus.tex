\documentclass{article}
% PACKAGES %
\usepackage[english]{} % Sets the language
\usepackage[margin=2cm]{geometry} % Sets the margin size
\usepackage{fancyhdr} % Allows creation of headers
\usepackage{xcolor} % Allows the use of color in text
\usepackage{float} % Allows figures and tables to be floats
\usepackage{appendix}
\usepackage{amsmath} % Enhanced math package prepared by the American Mathematical Society
	\DeclareMathOperator{\sech}{sech} % Include sech
\usepackage{amssymb} % AMS symbols package
\usepackage{mathrsfs}% More math symbols
\usepackage{bm} % Allows you to use \bm{} to make any symbol bold
\usepackage{bbold} % Allows more bold characters
\usepackage{verbatim} % Allows you to include code snippets
\usepackage{setspace} % Allows you to change the spacing between lines at different points in the document
\usepackage{parskip} % Allows you alter the spacing between paragraphs
\usepackage{multicol} % Allows text division into multiple columns
\usepackage{units} % Allows fractions to be expressed diagonally instead of vertically
\usepackage{booktabs,multirow,multirow} % Gives extra table functionality
\usepackage{hyperref} % Allows hyperlinks in the document
\usepackage{rotating} % Allows tables to be rotated
\usepackage{graphicx} % Enhanced package for including graphics/figures
	% Set path to figure image files
	\graphicspath{ {} }
\usepackage{listings} % for including text files
	\lstset{basicstyle=\ttfamily\scriptsize,
        		  keywordstyle=\color{blue}\ttfamily,
        	  	  stringstyle=\color{red}\ttfamily,
          	  commentstyle=\color{gray}\ttfamily,
          	 }		
\newcommand{\tab}{\-\hspace{1cm}}

% Create a header w/ Name & Date
\pagestyle{fancy}
\rhead{\textbf{Mitch Negus} \; 9/22/2017}

\begin{document}
\thispagestyle{empty}

{\bf {\large {NE250 Homework {1} \hfill Mitch Negus\\
		\hspace*{\fill} 9/22/2017\\ }}}
		
%%%%%%%%%%%%%%%%%%%%%%%%%%%%%%%%%% PROBLEM 1 %%%%%%%%%%%%%%%%%%%%%%%%%%%%%%%%%%

\section*{Problem 1}

The number of molecules, $N$, found in a sample of a compound with mass $M$ is 
$$ N(\cdot) = \frac{M(\cdot) N_A}{m(\cdot)} $$
where $m$ is the molar mass of the compound, and $N_A$ is Avogadro's number ($N_A = 6.022 \times 10^{23}$ molecules per mole). Where each actinide isotope is found only once in a molecule of its respective oxide, $N$ also gives the number of atoms of an isotope in the sample. 

To find $N$, we begin by finding the masses of the compounds in the fuel. We can decompose the total mass of the fuel, $M_f$, into its mixed oxide components:
$$ M_f = M(\text{UO}_2) + M(\text{PuO}_2). $$
Given a weight percent for plutonium, $w_{\text{P}}$, we note that $w_{\text{U}} = 1-w_{\text{P}}$ and so calculate the total masses of both the UO$_2$ and the PuO$_2$ to be
$$ M(\text{UO}_2) = (1-w_{\text{P}})M_f	$$
$$ M(\text{PuO}_2) = w_{\text{P}}M_f. $$

We are told that the uranium is all $^{238}$U, so 
$$ M(^{238}\text{UO}_2) = M(\text{UO}_2) = (1-w_{\text{P}})M_f. $$

The weight percents of the plutonium isotopes are also given,
$$ M(^{239}\text{PuO}_2) = w_{\text{P}9}M(\text{PuO}_2) = w_{\text{P}9}w_{\text{P}}M_f $$
$$ M(^{240}\text{PuO}_2) = w_{\text{P}0}M(\text{PuO}_2) = w_{\text{P}9}w_{\text{P}}M_f $$
$$ M(^{241}\text{PuO}_2) = w_{\text{P}1}M(\text{PuO}_2) = w_{\text{P}9}w_{\text{P}}M_f $$
$$ M(^{242}\text{PuO}_2) = w_{\text{P}2}M(\text{PuO}_2) = w_{\text{P}9}w_{\text{P}}M_f $$

Next, we must determine the molar masses of the various oxides. For uranium,
$$ m(^{238}\text{UO}_2) = m(^{238}\text{U}) + 2m(\text{O}), $$
and for plutonium,
$$ m(^{239}\text{PuO}_2) = m(^{239}\text{Pu}) + 2m(\text{O}) $$
$$ m(^{240}\text{PuO}_2) = m(^{240}\text{Pu}) + 2m(\text{O}) $$
$$ m(^{241}\text{PuO}_2) = m(^{241}\text{Pu}) + 2m(\text{O}) $$
$$ m(^{242}\text{PuO}_2) = m(^{242}\text{Pu}) + 2m(\text{O}) $$

Finally, we use both the total mass of a compound with its molar mass in the original formula, using provided or tabulated values ($w_{\text{P}} = 0.3, w_{\text{P}9}=0.697, w_{\text{P}0}=0.218, w_{\text{P}1}=0.058, w_{\text{P}2}=0.027, m(^{238}\text{U})=, m(^{239}\text{Pu})=, m(^{240}\text{Pu})=, m(^{241}\text{Pu})=, m(^{242}\text{Pu})=, m(\text{O})= $)


%%%%%%%%%%%%%%%%%%%%%%%%%%%%%%%%%% PROBLEM 2 %%%%%%%%%%%%%%%%%%%%%%%%%%%%%%%%%%

\section*{Problem 2}

We use the


\subsection*{\textit{a.})}





\end{document}







