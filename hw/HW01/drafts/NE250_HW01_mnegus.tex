\documentclass{article}
% PACKAGES %
\usepackage[english]{} % Sets the language
\usepackage[margin=2cm]{geometry} % Sets the margin size
\usepackage{fancyhdr} % Allows creation of headers
\usepackage{xcolor} % Allows the use of color in text
\usepackage{float} % Allows figures and tables to be floats
\usepackage{appendix}
\usepackage{amsmath} % Enhanced math package prepared by the American Mathematical Society
	\DeclareMathOperator{\sech}{sech} % Include sech
\usepackage{amssymb} % AMS symbols package
\usepackage{mathrsfs}% More math symbols
\usepackage{bm} % Allows you to use \bm{} to make any symbol bold
\usepackage{bbold} % Allows more bold characters
\usepackage{verbatim} % Allows you to include code snippets
\usepackage{setspace} % Allows you to change the spacing between lines at different points in the document
\usepackage{parskip} % Allows you alter the spacing between paragraphs
\usepackage{multicol} % Allows text division into multiple columns
\usepackage{units} % Allows fractions to be expressed diagonally instead of vertically
\usepackage{booktabs,multirow,multirow} % Gives extra table functionality
\usepackage[final]{pdfpages} % Allows pdfs to be imported
\usepackage{hyperref} % Allows hyperlinks in the document
\usepackage{rotating} % Allows tables to be rotated
\usepackage{graphicx} % Enhanced package for including graphics/figures
	% Set path to figure image files
	\graphicspath{ {fig/} }
\usepackage{listings} % for including text files
	\lstset{basicstyle=\ttfamily\scriptsize,
        		  keywordstyle=\color{blue}\ttfamily,
        	  	  stringstyle=\color{red}\ttfamily,
          	  commentstyle=\color{gray}\ttfamily,
          	 }		
\newcommand{\tab}{\-\hspace{1cm}}

% Create a header w/ Name & Date
\pagestyle{fancy}
\rhead{\textbf{Mitch Negus} \; 9/22/2017}

\begin{document}
\thispagestyle{empty}

{\bf {\large {NE250 Homework {1} \hfill Mitch Negus\\
		\hspace*{\fill} 9/22/2017\\ }}}
		
%%%%%%%%%%%%%%%%%%%%%%%%%%%%%%%%%% PROBLEM 1 %%%%%%%%%%%%%%%%%%%%%%%%%%%%%%%%%%

\section*{Problem 1}

The number of molecules, $N$, found in a sample of a compound with mass $M$ is 
$$ N(\cdot) = \frac{M(\cdot) N_A}{m(\cdot)} $$
where $m$ is the molar mass of the compound, and $N_A$ is Avogadro's number ($N_A = 6.022 \times 10^{23}$ molecules per mole). Where each actinide isotope is found only once in a molecule of its respective oxide, $N$ also gives the number of atoms of an isotope in the sample. 

To find $N$, we begin by finding the masses of the compounds in the fuel. We can decompose the total mass of the fuel, $M_f$, into its mixed oxide components:
$$ M_f = M(\text{UO}_2) + M(\text{PuO}_2). $$
Given a weight percent for plutonium, $w_{\text{P}}$, we note that $w_{\text{U}} = 1-w_{\text{P}}$ and so calculate the total masses of both the UO$_2$ and the PuO$_2$ to be
$$ M(\text{UO}_2) = (1-w_{\text{P}})M_f	$$
$$ M(\text{PuO}_2) = w_{\text{P}}M_f. $$

We are told that the uranium is all $^{238}$U, so 
$$ M(^{238}\text{UO}_2) = M(\text{UO}_2) = (1-w_{\text{P}})M_f. $$

<<<<<<< HEAD
The weight percents of the plutonium isotope oxides are also given (as a fraction of total plutonium oxide). Using $^{239}\text{PuO}_2$ as an example
$$ M(^{239}\text{PuO}_2) = w_{\text{P}9}M(\text{PuO}_2) = w_{\text{P}9}w_{\text{P}}M_f. $$


Next, we must determine the molar masses of the various oxides. For uranium,
$$ m(^{238}\text{UO}_2) = m(^{238}\text{U}) + 2m(\text{O}), $$
and similarly for plutonium.

When we combine the total and molar masses to determine the total number of atoms for each isotope, we find:\\
\tab 6.69$\times 10^{20} \text{ atoms of }^{238}$U	\\
\tab 4.65$\times 10^{20} \text{ atoms of }^{239}$Pu	\\
\tab 1.45$\times 10^{20} \text{ atoms of }^{240}$Pu	\\
\tab 3.84$\times 10^{19} \text{ atoms of }^{241}$Pu	\\
\tab 1.78$\times 10^{19} \text{ atoms of }^{242}$Pu	\\

(for full calculation, see Jupyter notebook, attached)

%%%%%%%%%%%%%%%%%%%%%%%%%%%%%%%%%% PROBLEM 2 %%%%%%%%%%%%%%%%%%%%%%%%%%%%%%%%%%

\section*{Problem 2}



%%%%%%%%%%%%%%%%%%%%%%%%%%%%%%%%%% PROBLEM 3 %%%%%%%%%%%%%%%%%%%%%%%%%%%%%%%%%%

\section*{Problem 3}

Cross sections plotted using ENDF/B-VII.1 from \href{http://atom.kaeri.re.kr/nuchart/}{KAERI}.

\begin{multicols}{2}
\tab\textbf{Fission Cross Sections}\\
\includegraphics[width=9cm]{fissionXS}

\tab\textbf{Capture Cross Sections}\\
\includegraphics[width=9cm]{captureXS}
\end{multicols}

Capture-to-fission ratios at:
\begin{multicols}{2}
\begin{onehalfspace}
\textbf{0.0253 eV}\\
$^{238}\text{U}$: \tab $\frac{3\text{ b}}{0.00003\text{ b}} = 100000$\\
$^{239}\text{Pu}$: \tab $\frac{300\text{ b}}{800\text{ b}} = 0.38$\\
$^{235}\text{U}$: \tab $\frac{100\text{ b}}{700\text{ b}} = 0.14$\\
\textbf{0.73 MeV}\\
$^{238}\text{U}$: \tab $\frac{0.13\text{ b}}{0.004\text{ b}} = 32.5$\\
$^{235}\text{U}$: \tab $\frac{0.13\text{ b}}{1\text{ b}} = 0.13$\\
$^{239}\text{Pu}$: \tab $\frac{0.07\text{ b}}{2\text{ b}} = 0.04$
\end{onehalfspace}
\end{multicols}



%%%%%%%%%%%%%%%%%%%%%%%%%%%%%%%%%% PROBLEM 9 %%%%%%%%%%%%%%%%%%%%%%%%%%%%%%%%%%

\section*{Problem 9}

\textbf{\textit{a.})} If rapidly compressed to half volume, the reactor's density will dramatically double. Cross sections vary directly with density, and so the reactor's cross sections will increase. Greater cross sections mean more chances for fission to be induced, and the multiplication factor will increase. A critical reactor would then become \underline{supercritical}.

\textbf{\textit{b.})} If squashed, into an ellipsoidal shape, the reactor's surface area would increase, with no change in volume. Greater surface area means that more neutrons would leak from the reactor, and the critical reactor would become \underline{subcritical} (the streaming term in the TE will grow, $k$ must decrease).

\textbf{\textit{c.})} Wrapping a sheet of cadmium around the outside of the reactor will cause neutrons to be reflected back into the reactor volume (assuming it replaces void). This will increase the neutrons able to cause fission, and the reactor will become \underline{supercritcal}  (the streaming term in the TE will diminish, $k$ must increase).

\textbf{\textit{d.})} Again, neutrons will be reflected back into the reactor. A greater neutron density increases the neutron flux and rate of fission. The multiplication factor will increase and the reactor will become \underline{supercritical}  (the streaming term in the TE will diminish, $k$ must increase).

\textbf{\textit{e.})} Adding an external neutron source to the reactor will also trigger an increase in the multiplication factor. There will be more neutrons in the system to cause fission events.

\textbf{\textit{f.})} Placing an identical reactor nearby the original could be considered as if adding another source. As in that case, the multiplication factor will increase.

\textbf{\textit{g.})} Over time, the fissionable nuclei in the reactor will be consumed in the fission reaction. With less fissionable nuclei, the fission cross sections of the reactor fuel material will decrease. The reactor will become \underline{subcritical} and the multiplication factor will decrease.



%%%%%%%%%%%%%%%%%%%%%%%%%%%%%%%%%% PROBLEM 10 %%%%%%%%%%%%%%%%%%%%%%%%%%%%%%%%%%

\section*{Problem 10}

\textbf{Assumptions of the neutron transport equation:}
\begin{enumerate}
	\item \textit{Neutrons are pointlike.} A reasonable assumption because the de Broglie wavelength of a neutron is significantlly less than the diameter of an atom. This assumption allows us to neglect rotation and quantum effects and write the transport equation as a function of energy. We need only being concerned with the particle's translational kinetic energy.
	\item \textit{Neutral particles travel in straight lines.} Neutron trajectories will not bend between collisions, and we can make the assumption that $\frac{\partial \theta}{\partial t} = 0$ and $\frac{\partial \varphi}{\partial t} = 0$.
	\item \textit{Particle-particle collisions are negligible.} Neutrons are generally very unlikely to collide/otherwise interact with other neutrons, allowing us to express the transport equation as a linear differential equation.
	\item \textit{Material properties are isotropic.} Valid when neutrons are moving with an appreciable velocity, this condition allows us to establish cross sections simply as functions of $\vec{r}$ and $E$ (not $\hat{\Omega}$.
	\item \textit{Material composition is independent of time.} Similar to assumption 5, our cross sections become dependent only on $\vec{r}$ and $E$. Though material does change over long time scales (\textit{i.e.} burnup) this assumption is valid for short term neutronics calculations.
	\item \textit{Quantities are expected values.} We may be unable to properly predict fluctuations in our results for cases where we are dealing with low density media.

\end{enumerate}



\includepdf[pages=-]{NE250_HW01_mnegus-notebook.pdf}



\end{document}







