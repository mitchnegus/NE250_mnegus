\documentclass{article}
% PACKAGES %
\usepackage[english]{} % Sets the language
\usepackage[margin=2cm]{geometry} % Sets the margin size
\usepackage{fancyhdr} % Allows creation of headers
\usepackage{xcolor} % Allows the use of color in text
\usepackage{float} % Allows figures and tables to be floats
\usepackage{appendix}
\usepackage{amsmath} % Enhanced math package prepared by the American Mathematical Society
	\DeclareMathOperator{\sech}{sech} % Include sech
\usepackage{amssymb} % AMS symbols package
\usepackage{mathrsfs}% More math symbols
\usepackage{breqn} % Allows line breaking in math mode
\usepackage{cancel} % Allows math strikethroughs to show cancellations
\usepackage{bm} % Allows you to use \bm{} to make any symbol bold
\usepackage{bbold} % Allows more bold characters
\usepackage{verbatim} % Allows you to include code snippets
\usepackage{setspace} % Allows you to change the spacing between lines at different points in the document
\usepackage{parskip} % Allows you alter the spacing between paragraphs
\usepackage{multicol} % Allows text division into multiple columns
\usepackage{units} % Allows fractions to be expressed diagonally instead of vertically
\usepackage{booktabs,multirow,multirow} % Gives extra table functionality
\usepackage[final]{pdfpages} % Allows pdfs to be imported
\usepackage{hyperref} % Allows hyperlinks in the document
\usepackage{rotating} % Allows tables to be rotated
\usepackage{graphicx} % Enhanced package for including graphics/figures
	% Set path to figure image files
	\graphicspath{ {fig/} }
\usepackage{listings} % for including text files
	\lstset{basicstyle=\ttfamily\scriptsize,
        		  keywordstyle=\color{blue}\ttfamily,
        	  	  stringstyle=\color{red}\ttfamily,
          	  commentstyle=\color{gray}\ttfamily,
          	 }		
\newcommand{\tab}{\-\hspace{1cm}}

\newcommand{\Oh}{\hat{\Omega}}
\newcommand{\cur}{\bm{J}}
\newcommand{\rt}{(\bm{r},t)}
\newcommand{\rOt}{(\bm{r},\Oh,t)}


% Create a header w/ Name & Date
\pagestyle{fancy}
\rhead{\textbf{Mitch Negus} \; 10/6/2017}

\begin{document}
\thispagestyle{empty}

{\bf {\large {NE250 Homework {2} \hfill Mitch Negus\\
		\hspace*{\fill} 10/6/2017\\ }}}
		
%%%%%%%%%%%%%%%%%%%%%%%%%%%%%%%%%% PROBLEM 1 %%%%%%%%%%%%%%%%%%%%%%%%%%%%%%%%%%

\section*{Problem 1}

If the energy distribution for fission neutrons from $^{235}\text{U}$ follows the functional approximation (for energy in MeV)
$$ \chi(E) = 0.453e^{-1.036E}\sinh(\sqrt{2.29E}), $$
then the most probable energy of a neutron corresponds to the maximum of the function. A maximum value will be found at a critical point of the function, which can be found via differentiation (specifically when $\frac{d\chi}{dE} = 0$):
\begin{align*}
\frac{d\chi}{dE}= 0	&= \frac{d}{dE}\left(0.453e^{-1.036E_{\text{max}}}\sinh(\sqrt{2.29E_{\text{max}}})\right) \\
				  0	&= 0.453\frac{d}{dE}\left(e^{-1.036E_{\text{max}}}\sinh(\sqrt{2.29E_{\text{max}}})\right) \\
				  0	&= 0.453\left[e^{-1.036E}\frac{d}{dE}\sinh(\sqrt{2.29E_{\text{max}}})+\frac{d}{dE}\left(e^{-1.036E_{\text{max}}}\right)\sinh(\sqrt{2.29E_{\text{max}}})\right] \\
				  0	&= 0.453\left[e^{-1.036E}\cosh(\sqrt{2.29E})\frac{d}{dE}\left(\sqrt{2.29E_{\text{max}}}\right)-1.036e^{-1.036E}\sinh(\sqrt{2.29E})\right] \\
				  0	&= 0.453\left[e^{-1.036E}\cosh(\sqrt{2.29E})\frac{\sqrt{2.29}}{2\sqrt{E_{\text{max}}}}-1.036e^{-1.036E}\sinh(\sqrt{2.29E})\right] \\
				  0	&= \frac{\sqrt{2.29}\cosh(\sqrt{2.29E})}{2\sqrt{E_{\text{max}}}}-1.036\sinh(\sqrt{2.29E}) \\
				  0	&= 1 - 1.369\sqrt{E_{\text{max}}}\tanh(\sqrt{2.29E_{\text{max}}}) \\
				  1	&= 1.369\sqrt{E_{\text{max}}}\tanh(\sqrt{2.29E_{\text{max}}}) \\
\end{align*}
$$\boxed{ E_{\text{max}}= 0.724\text{ MeV} }$$

The average energy can be found by finding the expected value of the function on the domain $[0,\infty)$.
\begin{align*}
E_{\text{ave}}	&= \int_0^{\infty} E\,\chi(E)\,dE \\
				&= \int_0^{\infty} E\left(0.453e^{-1.036E}\sinh(\sqrt{2.29E})\right)\,dE \\
				&= 0.453\int_0^{\infty} E\,e^{-1.036E}\sinh(\sqrt{2.29E})\,dE \\
\end{align*}
This integral cannot be solved analytically. Solving numerically (with Wolram Alpha),
$$\boxed{ E_{\text{ave}} = 1.98\text{ MeV} }$$



\newpage
%%%%%%%%%%%%%%%%%%%%%%%%%%%%%%%%%% PROBLEM 2 %%%%%%%%%%%%%%%%%%%%%%%%%%%%%%%%%%

\section*{Problem 2}

A general reaction rate for process $x$ as a function of energy can be defined as 
$$ R_x(E) = \Sigma_x(E) \phi(E) $$
where $\Sigma_x(E)$ is the macroscopic cross section for reaction $x$ and $\phi$ is the neutron flux, both at energy $E$. The macroscopic cross section can be further decomposed, so that
$$ R_x(E) = n_x \sigma_x(E) \phi(E) .$$
Comparing neutron-neutron reactions with all neutron-nuclei reactions, the ratio of reaction rates is
$$ \frac{R_{nn}(E)}{R_{tot}(E)} = \frac{n_n \sigma_{nn}(E) \phi(E)}{n_{\text{UO}_2} \sigma_{tot}(E) \phi(E)} .$$

We can know that the neutron flux is $\phi(0.025\text{ eV} = 10^{16}\text{ neutrons/(cm}^2\cdot\text{s)}$ and they are at thermal energies ($E = 0.025$ eV and traveling at $v = \sqrt{\frac{2(0.025\text{\tiny eV})}{m_n}} = 2.190\times10^5\text{ cm/s}$). The neutrons that are then in a 1 cm$^3$ volume at any given second is
$$ n_n = \frac{\phi(0.025\text{ eV})}{v} = \frac{10^{16}\text{ neutrons/(cm}^2\cdot\text{s)}}{2.190\times10^5\text{ cm/s}} = 4.566\times10^{10}\text{ neutrons/cm}^3 $$

For UO$_2$, $\rho = 10.97\text{ g/cm}^3$, $m_{\text{O}} = 16.0\text{ g/mol}$, $m_{\text{U}8} = 238.05\text{ g/mol}$, and $m_{\text{U}5} = 235.04\text{ g/mol}$. If we use an enrichment of 5\% (atom percent), then $m_{\text{UO}_2} = 0.95(238.05) + 0.05(235.04) + 2(16.0) = 269.90$ g/mol. For number density, we find
$$ n_{\text{UO}_2} = \frac{\rho N_A}{m_{\text{UO}_2}} = \frac{(10.97\text{ g/cm}^3)(6.022\times10^{23})}{269.90\text{ g/mol}} = 2.448\times10^{22}\text{ molecules/cm}^3$$

Additionally, we're given that $\sigma_{nn} = 10$ b, and we can determine the microscopic cross section for UO$_2$ from tabulated data. (From ENDF/B-VII.1 at 0.025 eV: $\sigma_{tot,\text{U}8} = 11.962\text{ b},\sigma_{tot,\text{U}5} = 698.856\text{ b}, \text{ and } \sigma_{tot,\text{O}} = 3.852\text{ b}$)
$$ \sigma_{tot} = 0.95\sigma_{tot,\text{U}8} + 0.05\sigma_{tot,\text{U}5} + 2\sigma_{tot,\text{O}} $$
$$ \sigma_{tot} = 0.95(11.962\text{ b}) + 0.05(698.856\text{ b}) + 2(3.852\text{ b}) $$
$$ \sigma_{tot} = 54.011\text{ b} $$

We can now sove for the ratio of reaction rates (noting that $\phi(E)$ cancels in the numerator and denominator)

$$ \frac{R_{nn}(E)}{R_{tot}(E)} = \frac{n_n \sigma_{nn}(E)}{n_{\text{UO}_2} \sigma_{tot}(E)} = \frac{(4.566\times10^{10}\text{ neutrons/cm}^3)(10\text{ b})}{(2.448\times10^{22}\text{ molecules/cm}^3)(54.011\text{ b})}$$
$$ \frac{R_{nn}(E)}{R_{tot}(E)} = 3.40\times10^{-13} $$
The rate of neutron-neutron collisions is \underline{13 orders of magnitudes less} than the rate of neutron-UO$_2$ collisions.



\newpage
%%%%%%%%%%%%%%%%%%%%%%%%%%%%%%%%%% PROBLEM 3 %%%%%%%%%%%%%%%%%%%%%%%%%%%%%%%%%%

\section*{Problem 3}



%%%%%%%%%%%%%%%%%%%%%%%%%%%%%%%%%% PROBLEM 4 %%%%%%%%%%%%%%%%%%%%%%%%%%%%%%%%%%

\section*{Problem 4}

First, we define the average scattering cosine $\bar{\mu}_0$ as the average dot product, $\langle \Oh \cdot \Oh' \rangle$. When normalized by $4\pi\Sigma_s$, the total of cross sections for scattering from any angle $\Oh$ to any other angle $\Oh'$, this is
$$ \bar{\mu}_0 \equiv \langle \Oh \cdot \Oh' \rangle = \left(\frac{1}{4\pi\Sigma_s}\right)  \int_{4\pi} d\Oh \int_{4\pi} d\Oh' \, \Oh \cdot \Oh' \Sigma_s(\Oh \cdot \Oh') $$

In the center of mass system, the probability that a particle scatters in any direction is roughly uniform, $\Sigma_{\text{CM}}(\theta_C) = \frac{\Sigma_s}{4\pi}$,
$$ \bar{\mu}_0 = \frac{1}{\Sigma_s}  \int_{4\pi} d\Oh \int_{4\pi} d\Oh' \, \Oh \cdot \Oh' \Sigma_{\text{CM}}(\theta_C) .$$




%%%%%%%%%%%%%%%%%%%%%%%%%%%%%%%%%% PROBLEM 5 %%%%%%%%%%%%%%%%%%%%%%%%%%%%%%%%%%

\section*{Problem 5}

A critical reactor has a multiplication factor of $k=1$. The multiplication factor can be definied as
$$ k \equiv \frac{\text{\# neutrons produced}}{\text{\# neutrons absorbed}} $$
Mathematically, the number of neutrons produced is $\int_0^E \nu \Sigma_f(E)\phi(E)\,dE$ and the number of neutrons absorbed is $\int_0^E \Sigma_a(E)\phi(E)\,dE$. Altogether, we can mathematically describe a critical reactor as 
$$ 1 = \frac{\int_0^E \nu \Sigma_f(E)\phi(E)\,dE}{\int_0^E \Sigma_a(E)\phi(E)\,dE} $$
or equivalently
$$ \int_0^E \nu \Sigma_f(E)\phi(E)\,dE = \int_0^E \Sigma_a(E)\phi(E)\,dE. $$
Since we are considering only thermal cross sections, we will let $\Sigma_X(E) = \Sigma_X(0.025\text{ eV}) = \Sigma_{X,T}$ and we find
$$ \nu \Sigma_{f,T} \int_0^E \phi(E)\,dE = \Sigma_{a,T} \int_0^E \phi(E)\,dE. $$
The integrals over flux cancel, and so
$$ \nu \Sigma_{f,T} = \Sigma_{a,T} .$$
The macroscopic cross sections can be rewritten as $\Sigma_{f,T} = \Sigma_{f,T,F}$ and $\Sigma_{a,T} = \Sigma_{a,T,F} + \Sigma_{a,T,M}$ where subscripts $F$ and $M$ denote fuel and moderator, respectively. Furthermore, each macroscopic cross section for each material can be expressed in terms of the material's number density and microscopic cross section, $\Sigma = n\sigma$. In total
$$ \nu n_F \sigma_{f,T,F} = n_F \sigma_{a,T,F} + n_M \sigma_{a,T,M} .$$
The fuel-to-moderator density at criticality can then be expressed as
$$ \frac{n_F}{n_M} = \frac{\sigma_{a,T,M}}{\nu \sigma_{f,T,F} - \sigma_{a,T,F}} .$$

(see Jupyter notebooks for full calculations)

\subsection*{\textit{a.}) \normalsize Graphite}
$$ \frac{n_F}{n_M} = 





%%%%%%%%%%%%%%%%%%%%%%%%%%%%%%%%%% PROBLEM 10 %%%%%%%%%%%%%%%%%%%%%%%%%%%%%%%%%%

\section*{Problem 10}

Answer the following questions as true or false, provide a one sentence justification for your answer\\
\begin{enumerate}
\item The integro-differential form of the transport equation expresses a local balance between neutron production and losses.
\item A vacuum boundary condition for the integro-differential transport equation implies a zero outgoing angular flux.
\item In the transport equation in curvilinear coordinates, the redistribution term allows neutrons to migrate between the directions Ωˆ as they move along a straight line.
\item A nuclear system is subcritical if it’s α eigenvalues satisfy max(Re(αj)) < 1.
\item The energy spectrum of the fundamental eigenmode of the α eigenvalue problem is skewed as though a 1 absorber is present.
\end{enumerate}

\end{document}







 
