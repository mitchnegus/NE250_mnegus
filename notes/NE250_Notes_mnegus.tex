\documentclass{report}
% PACKAGES %
\usepackage[english]{} % Sets the language
\usepackage[margin=2cm]{geometry} % Sets the margin size
\usepackage{fancyhdr} % Allows creation of headers
\usepackage{xcolor} % Allows the use of color in text
\usepackage{float} % Allows figures and tables to be floats
\usepackage{appendix}
\usepackage{amsmath} % Enhanced math package prepared by the American Mathematical Society
	\DeclareMathOperator{\sech}{sech} % Include sech
\usepackage{amssymb} % AMS symbols package
\usepackage{mathrsfs}% More math symbols
\usepackage{breqn} % Allows splitting of long math lines
\usepackage{bm} % Allows you to use \bm{} to make any symbol bold
\usepackage{bbold} % Allows more bold characters
\usepackage{verbatim} % Allows you to include code snippets
\usepackage{setspace} % Allows you to change the spacing between lines at different points in the document
\usepackage{parskip} % Allows you alter the spacing between paragraphs
\usepackage{multicol} % Allows text division into multiple columns
\usepackage{units} % Allows fractions to be expressed diagonally instead of vertically
\usepackage{booktabs,multirow,multirow} % Gives extra table functionality
\usepackage{hyperref} % Allows hyperlinks in the document
\usepackage{rotating} % Allows tables to be rotated
\usepackage{graphicx} % Enhanced package for including graphics/figures
	% Set path to figure image files
	\graphicspath{{fig/}}
\usepackage{listings} % for including text files
	\lstset{basicstyle=\ttfamily\scriptsize,
        		  keywordstyle=\color{blue}\ttfamily,
        	  	  stringstyle=\color{red}\ttfamily,
          	  commentstyle=\color{gray}\ttfamily,
          	 }		

% Shortcuts for common numbers
\newcommand{\tab}{\-\hspace{1cm}}
\newcommand{\h}[1]{\section*{#1}}
\newcommand{\hh}[1]{\subsection*{#1}}
\newcommand{\hhh}[1]{\subsubsection*{#1}}

\newcommand{\p}{\partial}
\newcommand{\ppt}{\frac{\p}{\p t}}
\newcommand{\grad}{\vec{\nabla}}

\newcommand{\Xs}{\Sigma}
\newcommand{\xs}{\sigma}

\newcommand{\Oov}{\frac{1}{v}}

\newcommand{\pos}{\vec{r}}
\newcommand{\cur}{\vec{J}}
\newcommand{\Oh}{\hat{\Omega}}

\newcommand{\intfp}{\int_{4\pi}}

\newcommand{\rt}{(\pos,t)}
\newcommand{\rEO}{(\pos,E,\Oh)}
\newcommand{\rOt}{(\pos,\Oh,t)}
\newcommand{\rOtprime}{(\pos,\Oh',t)}
\newcommand{\rEOt}{(\pos,E,\Oh,t)}
\newcommand{\rEOtprime}{(\pos,E',\Oh',t)}

\newcommand{\uthree}{^{233}\text{U}}
\newcommand{\ufive}{^{235}\text{U}}
\newcommand{\ueight}{^{238}\text{U}}
\newcommand{\punine}{^{239}\text{Pu}}
\newcommand{\puone}{^{241}\text{Pu}}


% Create a header w/ Name & Date
\pagestyle{fancy}
\rhead{\textbf{Mitch Negus} \; NE250 - Nuclear Reactor Theory}

\begin{document}
\thispagestyle{empty}

{\bf {\large {Nuclear Engineering 250 -- Nuclear Reactor Theory \hfill Mitch Negus}}}


\h{Nuclear Reactions}

Two types


\hh{1. Spontaneous \textsl{(decay)}}

\begin{itemize}
    \item $\alpha$: $^A_ZX \rightarrow ^{A-4}_{Z-2}X + ^4_2\alpha$
    \item $\beta$: $^A_ZX \rightarrow ^A_{Z+1}X + \beta + \bar{\nu}$
    \item $\gamma$: $X^* \rightarrow X + \gamma$
\end{itemize}

\hhh{Decay Equations}
\begin{multicols}{2}
$ t \rightarrow N(t) $\\
$ t+dt \rightarrow N(t+dt) $\\
$dN(t) = N(t+dt) - N(t) $\\
$dN(t) = -\lambda N(t) dt$ where $\lambda$ is the decay constant.
\end{multicols}
... working through, with B.C. $N(t=0) = N_0$ ...
$$N(t) = N_0 e^{-\lambda t}$$

\hhh{Mean Lifetime:}
$\frac{dN(t)}{N_0} = \lambda e^{-\lambda t} dt = p_d(t)$ where $p_d(t) dt$ is the probability of decay in time $dt$
$$\bar{t} = \int_0^{\infty} t p(t) dt = \frac{1}{\lambda}$$

\hhh{Half-Life:}
$T_{1/2}$ is defined as the time s.t. $N(T_{1/2}) = \frac{N_0}{2}$
$$T_{1/2} = \frac{ln2}{\lambda}$$


\hh{2. Induced \textsl{(projectile/target)}}
\textsl{This will be the emphasis of NE 250}

\hhh{ Neutron-Nucleus}

\begin{itemize}
    \item elastic scattering
    \item inelastic scattering/compound nuclear reaction  
    $$n + ^A_ZX \rightarrow ^{A+1}_ZX^*$$
	(could result in production of a $\gamma$ or $\alpha$, emission of a $n$, or fission)
	
	\textbf{Capture is a subset of absorption!}\\
	Absorption: $(n,\alpha), (n,\gamma), (n,f)$\\
	Capture: $(n,\gamma)$
\end{itemize}
	
\textbf{Cross Sections ($\xs$)}\\
- A property of the isotope and reaction\\
- A function of the isotope temperature (vibrational motion) and neutron speed (linear motion)\\
- Tabulated in XSec libraries \textbf{(3 common formats)}
\begin{itemize}
    \item ENDF (USA)
    \item JEFF (Europe/NEA)
    \item JENDL (Japan)
\end{itemize}

\-\\
\includegraphics[width=10cm]{Lecture01_U235-Xsec}
\-\\

\begin{itemize}
    \item Resonances in X-Sec plots due to excited energy levels that can be reached; nuclei only all excitation to these levels, and so only neutrons with this energy amount will be absorbed
    \item Cross sections measured at 300K (room temp); calculated using
	$$ E = k_BT $$ where $k_B$ is Boltzmann's constant and $T=300 \text{ K} \Rightarrow E = 0.0253 \text{ eV}$
    \item Higher temperatures cause resonant peak widths to broaden (less time spent near center of vibrational trajectory) $\rightarrow$ \textbf{Doppler Broadening}
\end{itemize}
\-\\
\textbf{Units}\\
\tab $1 \text{ barn} = 10^{24} \text{ cm}^2$ \\ 
\tab $1 \text{ eV} = 1.602 \times 10^{-19} \text{ J}$


\hh{Fission}

Can be spontaneous or induced\\
$$ n + \ufive \rightarrow X + Y + \bar{\nu} n + E $$
$\bar{\nu}$ is the average number of neutrons produced in a given fission event.\\
$ E_f \approx 200 \text{ MeV} $ (this is much higher than chemical reactions which are on the order of eV!)

\hhh{Fissile Isotopes}
$ E_b > E_{\text{threshold}}$\\

These neutrons could (almost) be considered as "able to fission from 0 KE neutrons."

Includes $\ufive$, $\uthree$, $\punine$, $\puone$

\hhh{Fissionable Isotopes}

Fission requires collision with high $E$ neutrons.

For $\ufive$ this is empirically given by $$\chi(E) = 0.453e^{-1.036E}\sinh(\sqrt{2.29E})$$

\includegraphics[width=10cm]{Lecture02_FissionEDist}


Also, note that $\bar{\nu}$ depends on the isotope. Below is a plot of $\bar{\nu}$ for $\punine$, $\uthree$ and $\ufive$:

\includegraphics[width=10cm]{Lecture02_nubarUPu}

\hhh{Fertile Isotopes}

Isotopes which either undergo neutron capture (and subsequent decay) to become fissile isotopes.


\hh{Energy breakdown of fission outputs}

\begin{itemize}
    \item ~ 180 MeV in the KE of fission products
    \item ~ 5 MeV in the kinetic energy of neutrons
    \item ~ 7 MeV in prompt $\gamma$s
    \item ~ 8 MeV in $\beta^{-}$ decay of fission products
    \item ~ 7 MeV in delayed $\gamma$s
    \item ~ 12 MeV in neutrinos
\end{itemize}

The energy from all outputs can be captured except for neutrinos.



\h{Criticality}


\hh{Multiplication Factor, $k$}

$$ k = \frac{\text{\# neutrons generated}}{\text{\# neutrons lost}} $$

$\text{\# neutrons generated = neutrons fission}$\\ 
$\text{\# neutrons lost = \# neutrons absorbed + \# neutrons leaked}$

\begin{itemize}
    \item $k=1$: the reaction is critical; the chain reaction is controlled (reactor)  
    \item $k<1$: the reactor is subcritcal; boring  
    \item $k>1$: the reaction is supercritical; this is a bomb  
\end{itemize}


\hh{Derivation of the Neutron Transport Equation}

Solving for the multiplication factor requires that we know: 

\begin{enumerate}
    \item $n$: neutron density $[n/\text{cm}^3]$
    \item $N$: atom/nuclide density $[\text{nuclei}/\text{cm}^3]$
    \item $\xs$: microscopic cross section $[\text{cm}^2]$
\end{enumerate}

\textbf{Reaction rate:} $ \; R = n N \xs$\\
\textbf{Macroscopic cross section $[1/\text{cm}]$:} $ \; \Xs = N \xs $\\
\textbf{Angular neutron density $[\frac{n}{\text{cm}^3 \cdot \text{sr}}]$:} $\; n\rEOt$\\

$\vec{v} = v \,\Oh, \; |\Oh| = 1$ (describes a sphere, formed by $\theta$ and $\phi$)\\
$ dr^3 = dx \; dy \; dz$\\
$ dE$\\
$ d\Oh = \sin\theta \; d\theta \; d\phi$; $d\Oh$ is a scalar, about the original position defined by vector $\Oh$.\\

\includegraphics[width=10cm]{Lecture02_PhaseSpace.png}

Altogether, $ n\rEOt \; d^3r \; dE \; d\Oh$, gives the \# of neutrons in the small volume about $\pos$ with energy $E$, and moving in direction $d\Oh$ about $\Oh$ at time t.

\textbf{Angular neutron flux (scalar):} $\phi\rEOt = v \; n\rEOt$\\
\textbf{Angular neturon current (vector):} $\cur\rEOt = \Oh \; \phi\rEOt$\\

We can find the number of neutrons in a volume $V$ using
$$ \int_V{ n\rEOt \; d^3r }$$

Change with time is then 
$$ \ppt\left[\int_V{ n\rEOt \; d^3r }\right] dE \; d\Oh = \text{ \# neutrons gained - \# neutrons lost}$$

\# neutrons gained: source (fission), in-scattering ($E', \Oh' \rightarrow E, \Oh$)\\
\# neutrons lost: absorption, scattering ($E, \Oh \rightarrow E', \Oh'$)\\

We also add a streaming term, to quantify neutrons leaking out (and in) to the system.

The chance of a collision in the system is given by 
$$ \left[ \int_V{ \Xs_{\text{tot}}(\pos,E) v n\rEOt \; d^3r} \right] dE \; d\Oh $$

The chance of fission in system... 
\-\\
\-\\
\-\\
\-\\
$$\text{(See paper notes...)}$$



\h{The Transport Equation}

\begin{dmath*}
\Oov \frac{\p\psi}{\p t} + \Oh \cdot \nabla\psi + \Xs_t\psi = \intfp d\Oh'\int_0^{\infty}dE' \Xs_s(E',\Oh'\rightarrow E,\Oh)\psi(E',\Oh') + \frac{\chi(E)}{4\pi}\int_0^{\infty}dE'\nu(E')\Xs_f(E') \intfp d\Oh'\phi\rEOtprime + s\rEOt
\end{dmath*}

\textbf{Initial condition:}
$\psi(\pos,E,\Oh,0) = \psi_0\rEO$

\textbf{Interface condition:}
angular flux must be continuous at all points

\hhh{Other conditions}
\tab \textbf{Fixed condition:} incoming flux is specified  
	$\psi(\pos_s,E,\Oh,t) = \psi_{\text{in}}\rEOt$\\
\tab\tab (Vacuum or black if $\psi_{\text{in}}\rEOt = 0$)\\
\tab \textbf{Reflective conditions:} mirror symmetry at some surface, $\psi(\Oh_{\text{in}},t) = \psi(\Oh_{\text{out}},t)$\\
\tab \textbf{Periodic conditions:} $\psi(\pos_s) = \psi(\pos_s + \vec{p})$\\
\tab \textbf{Finiteness conditions:} (can't have infinite flux)   
$0 \leq \psi\rEOt < \infty$\\
\tab \textbf{Source condition:} localized (pt.) sources introduce mathemetaical singularities\\
\tab\tab\tab\tab $S\rEOt = \lim_{\pos\rightarrow\vec{r_0}}\int dS \; \hat{e} \cdot \Oh$



\h{Approximations to the Transport Equation}


\hh{One-speed Transport Equation}

Assume all particles are the same speed: $\vec{v} = v_0 \cdot \Oh$

The equation becomes
$$ \Oov \frac{\p \psi\rOt}{\p t} + \Oh \nabla \psi \rOt + \Xs_t\psi \rOt = \intfp d\Oh' \Xs_s(\Oh'\rightarrow\Oh) \psi(\pos,\Oh',t) + \frac{\nu\Xs_f}{4\pi}\intfp d\Oh'\psi\rOtprime + S\rOt$$


\hh{One-dimensional}

$\pos = (x,y,z)$\\
$d\Oh = sin\theta \; d\theta \; d\varphi = d\mu \; d\varphi$, where $\mu = cos\theta$

$$ \Oov \frac{\p \psi(x,\Oh,t)}{\p t} + \Omega_x \frac{\p}{\p x} \psi (x,\Oh,t) + \Xs_t(x)\psi (x,\Oh,t) = \intfp d\Oh' \Xs_s(\Oh'\rightarrow\Oh) \psi(x,\Oh',t) + \frac{v\Xs_f}{4\pi}\intfp d\Oh' \psi(x,\Oh',t) + S(x,\Oh,t)$$



\h{The Diffusion Equation}

Usually the scalar flux is all that is needed to get a fairly accurate picture of our system. Reaction rates usually only depend on the neutron flux, not the direction of neutron motion.

\textbf{Assume} that the angular flux depends only weakly on direction.

\tab \textsl{Recall:}\\
\tab\tab\tab $\phi\rt = \intfp d\Oh \psi\rOt$\\
\tab\tab\tab $\cur\rt = \intfp d\Oh \; \Oh \;\psi\rOt$


\hh{The Neutron Continuity Equation}

\textbf{``The Zeroth Moment of the Transport Equation'':}\\
Integrate transport equation over all angles
\begin{dmath*}
\intfp d\Oh \left[\underbrace{\Oov\frac{\p \psi\rOt}{\p t}}_1 + \underbrace{\Oh \cdot \nabla \psi\rOt}_6 + \underbrace{\Xs_t \psi\rOt}_2 \right] = \intfp d\Oh \left[ \underbrace{\intfp d\Oh'\: \Xs_s(\Oh' \rightarrow \Oh) \psi\rOt}_5 + \underbrace{\frac{\nu \Xs_f}{4\pi} \intfp d\Oh'\: \psi\rOtprime}_3 + \underbrace{s\rOt}_4 \right]
\end{dmath*}

\begin{enumerate}
    \item \textbf{Time:} No approximations in time
$$ \Oov \ppt \intfp d\Oh \; \psi\rOt = \Oov \ppt \phi\rt $$

    \item \textbf{Absorption:} No approximations in absorption
$$\Xs_t \intfp d\Oh \; \psi\rOt = \Xs_t \; \phi\rt$$

    \item \textbf{Fission:} No approximations in fission
$$ \intfp d\Oh = \int_0^{\pi}\sin\theta \; d\theta \int_0^{2\pi} d\varphi = 4\pi$$
$$\intfp d\Oh \frac{\nu \Xs_f}{4\pi} \intfp d\Oh'\psi\rOtprime = \nu \Xs_f \phi\rt$$

    \item \textbf{Source:} No approximations in source 
$$\intfp d\Oh \; s\rOt \equiv S\rt$$

    \item \textbf{Scattering:} For scattering, interchange the order of integration
    $$ \intfp d\Oh \intfp d\Oh' \; \Xs_s(\Oh'\rightarrow \Oh) \; \psi\rOtprime = \intfp d\Oh' \intfp d\Oh \; \Xs_s(\Oh'\rightarrow\Oh) \; \psi\rOtprime$$

Now we assume that scattering is azimuthally symmetric (scattering depends only on cosine). The particle is as likely to scatter at angle $\theta$ in any direction off $ \Oh$.
    $$\intfp d\Oh \;\Xs_s(\Oh'\cdot\Oh) = 2\pi \int_{-1}^1 d\mu \; \Xs_s(\mu) = \Xs_s$$
Then, if we substitute this in above
    $$\Xs_s \intfp d\Oh' \; \psi\rOtprime =  \Xs_s \phi\rt$$

    \item \textbf{Streaming:} To adjust streaming, we first manipulate the order
$$ \intfp d\Oh \; \Oh \cdot \nabla \; \psi\rOt = \nabla \cdot \intfp d\Oh \; \Oh \; \psi\rOt = \nabla \cdot \cur\rt$$
\end{enumerate}

Put all the above integrations together to get the \textbf{neutron continuity equation (NCE)}. Notice that we have three equations (each $\pos$ has $x,y,z$ components) and 4 unknown quantities.

$$ \Oov \frac{\p \phi\rt}{\p t} + \nabla \cdot \cur\rt + \Xs_t \phi\rt = \Xs_s \phi\rt + \nu \Xs_f \phi\rt + S\rt $$


\hh{First angular moment}
Multiply the TE by $\Oh$ and integrate.  We will drop the fission term (technically we will assume it is part of the source), though the procedure is nearly the same when it is included.

Note:\\
\tab\tab $\intfp d\Oh \; \Oh = 0$\\
\tab\tab $\intfp d\Oh \; \Oh\Oh = \frac{4\pi}{3}\bar{\bar{I}}$, \tab $\bar{\bar{I}}$ is the identity tensor:  
	$$\intfp d\Oh \; \Oh_i\Oh_j = 
	\begin{cases}
	0, & i \neq j  \\
	\frac{4\pi}{3},&  i = j
	\end{cases}$$
\tab\tab $\intfp d\Oh \; \Oh\Oh\Oh = 0 $ 

When multiplied out, we have
$$ \intfp d\Oh \; \Oh\text{[TE]} $$
\begin{dmath*}
\underbrace{\intfp d\Oh \; \Oh \Oov \frac{\p \psi\rOt}{\p t}}_1 + \underbrace{\intfp d\Oh \; \Oh \Oh \nabla \psi \rOt}_5 + \underbrace{\intfp d\Oh \; \Oh \Xs_t\psi \rOt}_2 = \underbrace{\intfp d\Oh \; \Oh \intfp d\Oh' Xs_s(\Oh'\rightarrow\Oh) \psi(\pos,\Oh',t)}_4 + \underbrace{\intfp d\Oh \; \Oh \frac{\nu\Xs_f}{4\pi}\intfp d\Oh'\psi\rOtprime + \intfp d\Oh \; \Oh S\rOt}_3
\end{dmath*}

Like we did to derive the continuity equation, we can break down each term.
\begin{enumerate}
    \item \textbf{Time:}
    $$ \Oov \ppt \intfp d\Oh' \; \Oh\psi\rOt = \Oov \frac{\p\cur}{\p t} $$
    \item \textbf{Absorption:}
    $$ \Xs_t \intfp d\Oh \Oh\psi\rOt = \Xs_t \cur\rt $$
    \item \textbf{Source (fission now absent):}
    $$ \intfp d\Oh \; \Oh S\rOt \equiv S\rt $$
    \item \textbf{Scattering:} Expand scattering cross section in Legendre Polynomials (a sequence of orthogonal polynomials)
	$$P_n(x) = \frac{1}{2^n n!} \frac{d^n}{dx^n} \left[ (x^2 - 1)^n \right]$$
    Expand $\Xs_s(\Oh' \cdot \Oh)$ in Legendre polynomials
	$$ \Xs_s(\Oh' \cdot \Oh) = \sum_0^{\infty}{\frac{2\ell + 1}{4\pi}\Xs_{s\ell}P_{\ell}(\Oh')P_{\ell}(\Oh)} $$
	\begin{itemize}
	    \item $\ell = 0$ is isotropic\\ 
            \tab\tab\tab $P_0(\Oh) = 1 \quad\Rightarrow\quad \Xs_s(\Oh',\Oh) \approx \frac{1}{4\pi}\Xs_{s0}$
        \item $\ell = 1$ is linearly anisotropic\\  
            \tab\tab\tab $P_1(\Oh) = \Oh \quad\Rightarrow\quad \Xs_s(\Oh',\Oh) \approx \frac{1}{4\pi}(\Xs_{s0} + 3 \Oh' \cdot \Oh\Xs_{s1})$
            
            Assume scattering is at most linearly anisotropic (if its not, there's some "weird" stuff going on")
	\end{itemize}

	Substitute the linearly anistropic approximation into the expression for streaming. We note the prevously defined identities and defintion of neutron current:
	\begin{dmath*}
	\frac{1}{4\pi}\intfp d\Oh \Oh  \intfp d\Oh(\Xs_{s0} + 3 \Oh' \cdot \Oh\Xs_{s1}) \; \psi\rOt = \frac{1}{4\pi} \underbrace{\intfp d\Oh \; \Oh}_0 \intfp \Oh'\Xs_{s} \; \psi\rOtprime + \frac{1}{4\pi} \underbrace{\intfp d\Oh \; \Oh\Oh}_{4\pi/3\bar{\bar{I}}} \underbrace{\intfp d\Oh' \; \Oh' 3 \Xs_{s1} \; \psi\rOtprime}_{3\Xs_{s1}\cur\rt}
	\end{dmath*}
	
	Substituting those identities and simplifying, we get
	$$\left(\frac{1}{4\pi}\right) \left(\frac{4\pi}{3\bar{\bar{I}}}\right) \left(3\Xs_{s1}\cur\rt\right)= \Xs_{s1} \; J\rt$$


    \item \textbf{Streaming:}
	$$\intfp d\Oh \; \Oh \; \Oh \cdot \nabla \psi\rOt = \nabla \cdot \intfp d\Oh \; \Oh \;\Oh \; \psi\rOt $$

\end{enumerate}

Putting each of these 5 components back together the \textbf{current continuity equation (CCE)} is
$$\Oov \frac{\p \cur}{\p t} + \nabla \cdot \intfp d\Oh \;\Oh\Oh \; \psi\rOt + \Xs_t \cur\rt = \Xs_{s1}\cur\rt + S_1\rt$$

Now we have 2 moment equations (zeroth and first) so a total of 4 equations (neutron continuity and 3 tensor equations). There are now 10 unknowns–$\phi$ (1), $\cur$ (3), and the new tensor term (6). There is little point to continuing with this technique any longer.


\hh{Angular Flux Approximation}

Assume now that the flux is at most linearly anisotropic (for the current continuity equation we assumed that \textit{scattering} was linearly anisotropic).
$$ \psi(\Oh) \approx \frac{1}{4\pi}\left(\psi_0 + 3\Oh \cdot \vec{\psi_1}\right)$$

Substitute expansion into the streaming term of the current continuity equation.
$$\nabla \cdot \intfp d\Oh \;\Oh\Oh \; \psi\rOt = \nabla \cdot \intfp d\Oh \;\Oh\Oh \; \frac{1}{4\pi}\left(\psi_0 + 3\Oh \cdot \vec{\psi_1}\right)$$
$$ \nabla \cdot \frac{1}{4\pi} \int d\Oh \; \Oh\Oh \left(\psi_0 + 3 \Oh \vec{\psi_1}\right) = \nabla \cdot \frac{1}{4\pi} \left[ \intfp d\Oh \; \Oh\Oh \; \psi_0 + 3 \intfp d\Oh \; \Oh\Oh\Oh \; \vec{\psi}_1 \right] $$

Using the same identities as before, we have
$$ \nabla \cdot \frac{1}{4\pi} \int d\Oh \; \Oh\Oh \left(\psi_0 + 3 \Oh \vec{\psi_1}\right) = \nabla \cdot \frac{1}{4\pi} \frac{4\pi}{3}\bar{\bar{I}} \; \phi\rt $$
(Note, that this assumes $\psi_0 = \phi \leftarrow$ figure out where this comes from)
and
$$ \nabla \cdot \frac{1}{4\pi} \int d\Oh \; \Oh\Oh \left(\psi_0 + 3 \Oh \vec{\psi_1}\right) = \frac{1}{3} \nabla \phi\rt $$

Now the current continuity equation is 
$$\Oov \frac{\p \cur}{\p t} + \frac{1}{3} \nabla \phi\rt + \Xs_t \cur\rt = \Xs_{s1}\cur\rt + S_1\rt$$

Define absorption and transport cross sections:\\
$\Xs_a \equiv \Xs_t - \Xs_{s0}$\\
$\Xs_{tr} \equiv \Xs_t - \Xs_{s1}$\\

Using these cross sections we can reform both the neutron and current continuity equations.

\textbf{NCE:}
$$ \Oov \frac{\p \phi\rt}{\p t} + \nabla \cdot \cur\rt + \Xs_a \phi\rt = S\rt $$
\textbf{CCE:}
$$\Oov \frac{\p \cur}{\p t} + \frac{1}{3} \nabla \phi\rt + \Xs_{tr} \cur\rt = S_1\rt$$


\textbf{Fick's Law}\\

With the following conditions
\begin{itemize}
    \item Steady state $(\ppt(X) = 0)$
    \item Isotropic soure ($S_1 = 0$)
\end{itemize}

The current continuity equation becomes
$$\frac{1}{3} \nabla \phi(\pos) + \Xs_{tr} \cur(\pos) = 0$$
which we can solve for $\cur$:
$$\cur(\pos) = -\frac{1}{3\Xs_{tr}} \nabla \phi(\pos).$$
If we let $D = \frac{1}{3\Xs_{tr}} = \frac{1}{3(\Xs_t-\Xs_{s1})}$ be the diffusion coefficient, then this simplifies further to
$$\cur(\pos) = -D \nabla \phi(\pos)$$

------------------------------------------------------------------\\
Recall:\\
$\Xs_{s1} = \int d\Oh \; \Oh \Xs_{s}$\\
Include azimuthally symmetric assumptions and $\Xs_{s1} = \bar{\mu}_0 \Xs_s$, where $\bar{\mu}_0$ = average scattering cosine = $\frac{2}{3A}$, where $A$ is the atomic mass number.

Then $\Xs_{tr} = \Xs_t - \bar{\mu}_0 \Xs_s$.\\
------------------------------------------------------------------\\

From this implementation of Fick's law, we can write the diffusion equation as only a function of $\phi$:

$$\Oov \ppt \phi\rt - \nabla \cdot D \nabla \phi\rt + \Xs_a \phi\rt = \nu\Xs_f \phi\rt + S\rt$$

Due to our assumptions, however, the diffusion equation is not valid at 
\begin{enumerate}
    \item Boundaries/interfaces
    \item Sources
    \item Strong absorbers
    \item Voids
\end{enumerate}

The angular flux expansion can then be written as 
$$\psi\rOt \approx \frac{1}{4\pi}\left( \phi\rt - \frac{1}{\Xs_{tr}} \nabla \phi\rt \right)$$

At equilibrium, $\cur = 0$ (net migration is opposed to gradient, hence negative into Fick's Law)

\textbf{Mean Free Path:} the median distance from the last collision\\
$\lambda_t = \frac{1}{\Xs_t}$ or $\lambda_{tr} = \frac{1}{\Xs_{tr}}$

If scattering is 
\begin{itemize}
    \item isotropic $\rightarrow \bar{\mu}_0 = 0, \lambda_t = \lambda_{tr}$
    \item forward peaked $\rightarrow \bar{\mu}_0 > 0, \lambda_t > \lambda_{tr}$
\end{itemize}

\textbf{Initial Conditions:} $\phi(\pos,0) = \phi(\pos) \; \forall \pos \in V$

Basic requirements:\\
- real and nongegative ($\phi \geq 0$)\\
- bounded ($\phi > \infty$) \\

\textbf{Interface conditions:} \\
-Zeroth and First moments must be continuous\\
\tab$\phi_1\rt = \phi_2\rt$\\
\tab$\cur_1\rt = \cur_2\rt$\\

\textbf{Vacuum BCs:}\\
%$\psi\rsOt = \Oh \hat{e}_s < 0$\\
In the DE use partial current\\
\tab $J_{-}\rt = \int_{2\pi^{-}} d\Oh \Oh \cdot \hat{e}_s \psi\rOt = 0$\\
\tab $J_{+}\rt = \int_{2\pi^{+}} d\Oh \Oh \cdot \hat{e}_s \psi\rOt = 0$\\

$J_{\mp} = \int_{2\pi^{-}} d\Oh \Oh \hat{e}_s \left( \frac{1}{4\pi} \left( \phi \pm \frac{1}{\Xs_{tr}} \nabla \phi \right)\right) \approx \frac{1}{4}\phi \pm \frac{D}{2}\hat{e}_s \cdot \nabla \phi = 0$

In 1D problems
$J_{-}(\pos_s,t) = J_{-}(x_s,t) = \frac{1}{4}\phi(x_s,t) + \frac{D}{2}\frac{d\phi}{dx}|_{x_s} = 0$

The extrapolation distance, where $\phi = 0$ is
$$\tilde{x}_s = x_s + 2D = x_s + \frac{2}{3}\lambda_{tr}$$
and so replace $J_{-}(x_s) = 0$ with $ \phi(\tilde{x}_s) = 0$.






\end{document}




