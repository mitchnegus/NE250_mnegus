\documentclass{report}
% PACKAGES %
\usepackage[english]{} % Sets the language
\usepackage[margin=2cm]{geometry} % Sets the margin size
\usepackage{fancyhdr} % Allows creation of headers
\usepackage{xcolor} % Allows the use of color in text
\usepackage{float} % Allows figures and tables to be floats
\usepackage{appendix}
\usepackage{amsmath} % Enhanced math package prepared by the American Mathematical Society
	\DeclareMathOperator{\sech}{sech} % Include sech
\usepackage{amssymb} % AMS symbols package
\usepackage{mathrsfs}% More math symbols
\usepackage{bm} % Allows you to use \bm{} to make any symbol bold
\usepackage{bbold} % Allows more bold characters
\usepackage{verbatim} % Allows you to include code snippets
\usepackage{setspace} % Allows you to change the spacing between lines at different points in the document
\usepackage{parskip} % Allows you alter the spacing between paragraphs
\usepackage{multicol} % Allows text division into multiple columns
\usepackage{units} % Allows fractions to be expressed diagonally instead of vertically
\usepackage{booktabs,multirow,multirow} % Gives extra table functionality
\usepackage{hyperref} % Allows hyperlinks in the document
\usepackage{rotating} % Allows tables to be rotated
\usepackage{graphicx} % Enhanced package for including graphics/figures
	% Set path to figure image files
	%\graphicspath{ } }
\usepackage{listings} % for including text files
	\lstset{basicstyle=\ttfamily\scriptsize,
        		  keywordstyle=\color{blue}\ttfamily,
        	  	  stringstyle=\color{red}\ttfamily,
          	  commentstyle=\color{gray}\ttfamily,
          	 }		
\newcommand{\tab}{\-\hspace{1cm}}

% Create a header w/ Name & Date
\pagestyle{fancy}
\rhead{\textbf{Mitch Negus} \; 9/15/2017}

\begin{document}
\thispagestyle{empty}

{\bf {\large {NE250 Summary {3} \hfill Mitch Negus\\
		\hspace*{\fill} 9/15/2017\\ }}}
\section*{\textsl{Molten salt reactors: A new beginning for an old idea} \\ \normalsize David LeBlanc}

\subsection*{Notes}

Advantages of MSRs over solid-fuel designs\\
-fluid fuel $\Rightarrow$ no meltdown (instead, drain to passively cooled containment)\\
-fission products form stable fluorides (stay w/in salt during accident) or are volatile (continuous removal already will take place)\\
-noble gases bubble out\\
-metal plates out or can be collected (metal sponges)\\
-removal of xenon eliminates dead time required in traditional shut-down\\
-large temp coeffs\\
-low pressure vessel\\
-no water/sodium (no explosions)\\
-fissile concentrations can be adjusted as needed\\
-U233/Th produces less transuranic waste (few hundred years to less than U ore levels of radiotoxicity)\\
-can process conventional nuclear waste

Oak Ridge MSBR (Molten Salt Breeder Reactor)\\
-Developed in late 1960s\\
-priority was on reducing doubling time\\
\tab - nuclear power expected to grow exponentially\\
\tab - current uranium reserves were insufficient\\
-power station designs began as sphere w/in sphere (fuel=inner;blanket=outer)\\
-became graphite separator (between fuel and blanket) made replacement of reactor components excessively costly/difficult\\
-$^{233}\text{UF}_4$ in fuel salt; $\text{ThF}_4$in blanket salt\\
-bubbling fluorine gas through fuel salt converts $\text{UF}_4$ to $\text{UF}_6$ (volatile; easily separated) $\rightarrow$ established procedure for converting $\text{UF}_6$ to $\text{UF}_4$\\
-Liquid Bismuth Reductive Extraction for removing FPs\\
-final design had outside undermoderated zone for better neutron absorption; still graphite moderated (including graphite pebbles); was single fluid\\

More recent priorities have emerged of MSR program\\
-Net power/cost\\
-Long-lived waste reduction\\
-Safety\\
-Proliferation resistance\\
-Resource utilization\\
All can be well addressed (to varying degrees) by variations on the MSR technology and fuel cycle\\

Proposals to solve two-fluid problem ("plumbing problem"--need to replace graphite reactor vessel as a whole unit)\\
-advantage of two-fluid is extraction of FPs\\
-disadvantage was small critical diameter (~1m) $\rightarrow$ small core = limited power output\\
-plumbing, allowing mixing of two fluids solved power output problem, makes graphite components excessively complicated\\
-solution is to use long cylinder (as opposed to spheres or short cylinder)\\

Designs with graphite moderator\\
-likely require multiple cylindrical core units to reach GW scale
\tab - not too problematic; graphite barriers likely will need replacement, more cores make this easier\\
-require 150-400kg/GW(e) startup fissile inventory (previous ORNL designs required 700kg/GW(e) for 2-fluid, 1500kg/GW(e) for 1-fluid; LWR requires 3000-5000kg/GW(e))\\

Designs without graphite moderator\\
-GW single cores attainable (though benefits still exist for having more cores)\\
-carrier salt serves as decent moderator\\
-simple design/small size(6.6m length $\times$ 0.7m diameter--can be fit onto tractor trailer for transport) $\rightarrow$ 224MWe reactor\\

Fuel\\
-current designs would utilize $^{233}\text{U}$; no abundant source exists\\
-could operate on HEU; public perception/proliferation concerns\\
-option to start on LEU exists if one starts on LEU but does not extract $^{233}\text{U}$\\

DMSR Converter\\
-ONRL study; single fluid, 30 yr once through, 1000MW(e), startup w/ 20\% LEU\\
-lifetime uranium utilization = 1810 tonnes (oppsoed to 6400 tonnes for LWR)\\
-many further modifications possible (i.e. pebble bed graphite mod)\\

Three suggested 200MW(e) designs\\
\begin{enumerate}
    \item Th-$^{233}\text{U}$ cycle\\
            two-fluid\\
            no graphite (metal tube)\\
            800 kg/GW(e) startup fissile inventory\\
            6 month FP processing cycle
            4m length $\times$ 0.85m diameter dimensions
    \item LEU+Th converter reactor\\
            single-fluid (similar to ORNL design\\
            12-15 year graphite replacement\\
            salt replacement/transuranic removal at same time)\\
            4m $\times$ 4m max dimensions\\
    \item Pebble bed moderator design\\
            undermoderated outer zone\\
            starting fissile inventory\\
            ... length $\times$ 4.4m diameter max dimensions
\end{enumerate}

***check "As a first approximation the critical diameter will be the ratio of the Buckling constants between the given geometries. Thus, for the same fuel salt and/or graphite combination, a long cylinder will have a critical diameter approximately 77\% that of a sphere. If a specific combination of fissile concentration, graphite and carrier salt gives a critical diameter of 1 m for a sphere, then a long cylinder would have critical diameter of 0.77 m."






\end{document}




