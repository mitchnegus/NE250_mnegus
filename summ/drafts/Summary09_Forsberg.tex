\documentclass{report}
% PACKAGES %
\usepackage[english]{} % Sets the language
\usepackage[margin=2cm]{geometry} % Sets the margin size
\usepackage{fancyhdr} % Allows creation of headers
\usepackage{xcolor} % Allows the use of color in text
\usepackage{float} % Allows figures and tables to be floats
\usepackage{appendix}
\usepackage{amsmath} % Enhanced math package prepared by the American Mathematical Society
	\DeclareMathOperator{\sech}{sech} % Include sech
\usepackage{amssymb} % AMS symbols package
\usepackage{mathrsfs}% More math symbols
\usepackage{bm} % Allows you to use \bm{} to make any symbol bold
\usepackage{bbold} % Allows more bold characters
\usepackage{verbatim} % Allows you to include code snippets
\usepackage{setspace} % Allows you to change the spacing between lines at different points in the document
\usepackage{parskip} % Allows you alter the spacing between paragraphs
\usepackage{multicol} % Allows text division into multiple columns
\usepackage{units} % Allows fractions to be expressed diagonally instead of vertically
\usepackage{booktabs,multirow,multirow} % Gives extra table functionality
\usepackage{hyperref} % Allows hyperlinks in the document
\usepackage{rotating} % Allows tables to be rotated
\usepackage{graphicx} % Enhanced package for including graphics/figures
	% Set path to figure image files
	%\graphicspath{ } }
\usepackage{listings} % for including text files
	\lstset{basicstyle=\ttfamily\scriptsize,
        		  keywordstyle=\color{blue}\ttfamily,
        	  	  stringstyle=\color{red}\ttfamily,
          	  commentstyle=\color{gray}\ttfamily,
          	 }		
\newcommand{\tab}{\-\hspace{1cm}}

% Create a header w/ Name & Date
\pagestyle{fancy}
\rhead{\textbf{Mitch Negus} \; 10/29/2017}

\begin{document}
\thispagestyle{empty}

{\bf {\large {NE250 Summary {9} \hfill Mitch Negus\\
		\hspace*{\fill} 10/29/2017\\ }}}
\section*{\textsl{Strategies for a Low-Carbon Electricity Grid With Full Use of Nuclear, Wind and Solar Capacity to Minimize Total Costs} \\ \normalsize C. Forsberg}

\tab ... \\
\tab ...



\end{document}

-Discusses potential of 5 classes of systems to match electric production to demand
	-electrics (batteries)
	-FIRES (firebrick, industrial heat applications)
	-RATHS (includes thermal heat storage through salts)
	-NUTOC (nuclear with topped off peak electricity)
	-hybrids (electricity to grid and second product when prices low)
-Current demand curve bell shaped (fossil fuel dominated)
	-low electricity demand, only cheapest plants are used, prices low
	-high electricity demand, more expensive plants brought online, prices high
	-prices set by highest ``cost'' plant
-Renewable/nuclear demand curve bimodal
	-renewables provide very cheap power when generating
		-producers bid down prices to continue sales
	-power pricing increases dramatically when inactive
-Significant use of renewables forces price collapse
	-energy price in solar/wind dominated market is cheap when there is sun/wind
	-price of energy increases when supply diminishes
	-price collapse impacts most severe for causal energy (most synchronous)
-Price collapsing forces other generators to compensate (at a loss if price does not increase to cover losses during collapsed price period)
-Subsidies for price collapse do not solve problem -> pass it along (to taxpayers)

Analysis of Electricity Matching System Classes
-electrics are economically best for frequent charge/discharge cycles 
	-day/night solar cycles, not week/month weather patterns
	-many cycles per year required for low levelized costs
-FIRES limits price collapse
	-buy electricity for industrial process heat at low energy prices
	-becomes extremely competitive if renewables reach cost of fossil fuels
	-allows coupling of electric/industrial sector decarbonization
-RATHs provides longer term (week/month/seasonal) storage alternative
	-preferentially couple to nuclear plants
		-many cycle recharge still better for maximizing levelized costs
		-nuclear plants assure recharge (cloudy weather limits solar thermal)



