\documentclass{report}
% PACKAGES %
\usepackage[english]{} % Sets the language
\usepackage[margin=2cm]{geometry} % Sets the margin size
\usepackage{fancyhdr} % Allows creation of headers
\usepackage{xcolor} % Allows the use of color in text
\usepackage{float} % Allows figures and tables to be floats
\usepackage{appendix}
\usepackage{amsmath} % Enhanced math package prepared by the American Mathematical Society
	\DeclareMathOperator{\sech}{sech} % Include sech
\usepackage{amssymb} % AMS symbols package
\usepackage{mathrsfs}% More math symbols
\usepackage{bm} % Allows you to use \bm{} to make any symbol bold
\usepackage{bbold} % Allows more bold characters
\usepackage{verbatim} % Allows you to include code snippets
\usepackage{setspace} % Allows you to change the spacing between lines at different points in the document
\usepackage{parskip} % Allows you alter the spacing between paragraphs
\usepackage{multicol} % Allows text division into multiple columns
\usepackage{units} % Allows fractions to be expressed diagonally instead of vertically
\usepackage{booktabs,multirow,multirow} % Gives extra table functionality
\usepackage{hyperref} % Allows hyperlinks in the document
\usepackage{rotating} % Allows tables to be rotated
\usepackage{graphicx} % Enhanced package for including graphics/figures
	% Set path to figure image files
	%\graphicspath{ } }
\usepackage{listings} % for including text files
	\lstset{basicstyle=\ttfamily\scriptsize,
        		  keywordstyle=\color{blue}\ttfamily,
        	  	  stringstyle=\color{red}\ttfamily,
          	  commentstyle=\color{gray}\ttfamily,
          	 }		
\newcommand{\tab}{\-\hspace{1cm}}

% Create a header w/ Name & Date
\pagestyle{fancy}
\rhead{\textbf{Mitch Negus} \; 9/22/2017}

\begin{document}
\thispagestyle{empty}

{\bf {\large {NE250 Summary {4} \hfill Mitch Negus\\
		\hspace*{\fill} 9/22/2017\\ }}}
\section*{\textsl{A Global Physics Approach to Transmutation of Radioactive Nuclei} \\ \normalsize M. Salvatores, I. Slessarev, and M. Uematsu}

\tab There are many proposed concepts for the transmuation of radioactive nuclei, especially those with long half-lifes and highly energetic emissions produced by fission reactions, into shorter-lived, less dangerous products. This article constructs a mathematical analysis of the transmutation process, and uses this analysis to evaluate a series of different reactor technologies and their potential to provide radiologically clean nuclear power (RCNP). The analysis begins by developing a series of rate equations based on a materials isotopic composition, and the tendency of those isotopes to produce or absorb neutrons in the process of transmutation into short-lived nuclei (SLN). Once this structure has been developed, the authors compare the transmutation potential of a variety of different reactor types: LWRs, superthermal spectrum reactors, intermediate spectrum steam-water reactors (SWRs),and fast reactors. An additional discussion is provided on insight into how hybrid accelerator systems could contribute to a RCNP solution.\\
\tab At the end of the article, I thought the final discussion of which systems actually could be implemented to achieve RCNP was useful for providing context to the work. It was interesting, though somewhat discouraging, to see that without isotopic separation, the authors found no solution to completely burn the waste from current LWRs. While I believe it would have been outside the scope of this paper, I think it would be interesting to make some analysis of how the theoretically most useful technologies for transmutation (fast reactors and hybrids) compare in terms of technological readiness.

\end{document}







