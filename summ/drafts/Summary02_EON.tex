\documentclass{report}
% PACKAGES %
\usepackage[english]{} % Sets the language
\usepackage[margin=2cm]{geometry} % Sets the margin size
\usepackage{fancyhdr} % Allows creation of headers
\usepackage{xcolor} % Allows the use of color in text
\usepackage{float} % Allows figures and tables to be floats
\usepackage{appendix}
\usepackage{amsmath} % Enhanced math package prepared by the American Mathematical Society
	\DeclareMathOperator{\sech}{sech} % Include sech
\usepackage{amssymb} % AMS symbols package
\usepackage{mathrsfs}% More math symbols
\usepackage{bm} % Allows you to use \bm{} to make any symbol bold
\usepackage{bbold} % Allows more bold characters
\usepackage{verbatim} % Allows you to include code snippets
\usepackage{setspace} % Allows you to change the spacing between lines at different points in the document
\usepackage{parskip} % Allows you alter the spacing between paragraphs
\usepackage{multicol} % Allows text division into multiple columns
\usepackage{units} % Allows fractions to be expressed diagonally instead of vertically
\usepackage{booktabs,multirow,multirow} % Gives extra table functionality
\usepackage{hyperref} % Allows hyperlinks in the document
\usepackage{rotating} % Allows tables to be rotated
\usepackage{graphicx} % Enhanced package for including graphics/figures
	% Set path to figure image files
	%\graphicspath{ } }
\usepackage{listings} % for including text files
	\lstset{basicstyle=\ttfamily\scriptsize,
        		  keywordstyle=\color{blue}\ttfamily,
        	  	  stringstyle=\color{red}\ttfamily,
          	  commentstyle=\color{gray}\ttfamily,
          	 }		
\newcommand{\tab}{\-\hspace{1cm}}

% Create a header w/ Name & Date
\pagestyle{fancy}
\rhead{\textbf{Mitch Negus} \; 9/8/2017}

\begin{document}
\thispagestyle{empty}

{\bf {\large {NE250 Summary {2} \hfill Mitch Negus\\
		\hspace*{\fill} 9/8/2017\\ }}}
\section*{\textsl{What Will Advanced Nuclear Power Plants Cost?} \\ \normalsize Energy Options Network (EON)}

\tab The tremendous up front costs of traditional nuclear designs are arguably the steepest barrier facing the widespread adoption of nuclear power. Advanced nuclear designs are being pursued with this challenge in mind, and the various companies and startups behind those designs contend that their designs are less costly than conventional plants. Unfortunately, it can be difficult to assess the cost projections of companies who wish to keep many aspects of their designs proprietary. The Energy Options Network (EON) study is an effort to accurately assess the estimated costs of advanced nuclear projects, allowing internal comparisons between projects to be drawn in addition to enabling analyses between advanced nuclear concepts and either conventional nuclear power or other existing energy sources.\\
\tab The study focused on eight companies pursuing advanced reactor designs with capacities greater than 30 MW. These were Elysium Industries, GE, Moltex Energy, NuScale Power, Terrestrial Energy, ThorCon Power, Transatomic Power, and X-energy. Using cost-categories developed by the Gen IV International Forum, the EON was able to collect cost estimates in a consistent format between each company. For each cost-category and their various subcategories, default values were provided and companies were asked to provide an "adjustment multiplier" for how their estimate deviated from the default. Default values were selected based on previous studies of advanced reactors. Notably, all companies which participated in the study had lower capital cost, operating cost, and levelized-cost-of-electricity (LCOE) projections than traditional nuclear plants. This is somewhat expected, however, if advanced nuclear plants are aiming to tackle the challenge of constructing economically viable products. More specifically, while not all companies saw savings in direct construction costs, all saw substantial reductions in indirect services costs.\\
\tab Personally, I found the study refreshing. The potential of advanced nuclear is frequently touted, and many of the involved companies are vocal proponents of their designs. Still, many refuse to provide details. This lack of transparency is understandable for protecting proprietary information, but also hurts my ability to evaluate the company's credibility. Simply by participating in this study, even where readers can't see individual cost estimates, boosts my perception of an individual company's credibility, in addition to the advanced nuclear industry in general. Also, I was initially surprised by the methodology of using an adjustment multiplier, but after consideration, I thought it was a good solution. It is a tool that allows non-experts to appreciate when a cost-category is many times cheaper than the standard rather than simply providing a cost. I have no idea how much a generator costs off the top of my head, but I can be skeptical if a company says they have a generator that costs 100 times less than the conventional estimate. I do wish that the study had gone into a deeper breakdown of data, or at least provided the study's data for subcategories. I think the provision of simple distributions of subcategory predictions (\textit{i.e.} generators had a mean projected value of $x$ with a st. dev of $\sigma$) would be insightful.\\



\end{document}

-Study undertaken to accurately estimate costs for advanced nuclear designs\\
-Difficult to evaluate costs of adv. nuclear when companies hold proprietary info\\
-This study allowed a third party to aggregate cost estimates in a consistent format between groups\\
-Elysium, GE, Transatomic, Moltex, NuScale, Terrestrial Energy, ThorCon, X-Energy\\
-Study used Gen IV forum accounting protocols (follows DOE EEDB)\\
-Coarse categories, broken down smaller\\
-Defaults were provided, companies asked to denote deviations\\
-All participating companies had lower capital, operating, and LCOE projections than current nuclear designs (this is somewhat expected... if not, why do this?)\\
-Not all lower direct construction costs, but all saw substantial reductions in indirect services\\

-I liked the aggregated discussion, the study actually gave credibility to the companies, not just the industry\\
-I was surprised by the methodology (giving deviation from default) but thought it was a good solution; it prompts self-evaluation (non-experts can appreciate when something is 100x cheaper than the standard rather than saying our new pumps will cost \$50,000?when the going rate is \$5M)\\
-I wish they had gone into a deeper breakdown of data (i.e. distributions of predictions at a more specific level like turbine generators had a mean projected value of $x$ with a st. dev of $\sigma$)\\




