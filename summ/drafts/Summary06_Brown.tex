\documentclass{report}
% PACKAGES %
\usepackage[english]{} % Sets the language
\usepackage[margin=2cm]{geometry} % Sets the margin size
\usepackage{fancyhdr} % Allows creation of headers
\usepackage{xcolor} % Allows the use of color in text
\usepackage{float} % Allows figures and tables to be floats
\usepackage{appendix}
\usepackage{amsmath} % Enhanced math package prepared by the American Mathematical Society
	\DeclareMathOperator{\sech}{sech} % Include sech
\usepackage{amssymb} % AMS symbols package
\usepackage{mathrsfs}% More math symbols
\usepackage{bm} % Allows you to use \bm{} to make any symbol bold
\usepackage{bbold} % Allows more bold characters
\usepackage{verbatim} % Allows you to include code snippets
\usepackage{setspace} % Allows you to change the spacing between lines at different points in the document
\usepackage{parskip} % Allows you alter the spacing between paragraphs
\usepackage{multicol} % Allows text division into multiple columns
\usepackage{units} % Allows fractions to be expressed diagonally instead of vertically
\usepackage{booktabs,multirow,multirow} % Gives extra table functionality
\usepackage{hyperref} % Allows hyperlinks in the document
\usepackage{rotating} % Allows tables to be rotated
\usepackage{graphicx} % Enhanced package for including graphics/figures
	% Set path to figure image files
	%\graphicspath{ } }
\usepackage{listings} % for including text files
	\lstset{basicstyle=\ttfamily\scriptsize,
        		  keywordstyle=\color{blue}\ttfamily,
        	  	  stringstyle=\color{red}\ttfamily,
          	  commentstyle=\color{gray}\ttfamily,
          	 }		
\newcommand{\tab}{\-\hspace{1cm}}

% Create a header w/ Name & Date
\pagestyle{fancy}
\rhead{\textbf{Mitch Negus} \; 10/6/2017}

\begin{document}
\thispagestyle{empty}

{\bf {\large {NE250 Summary {6} \hfill Mitch Negus\\
		\hspace*{\fill} 10/6/2017\\ }}}
\section*{\textsl{Neutronic performance of uranium nitride composite fuels in a PWR} \\ \normalsize N. Brown, A. Aronson, M. Todosow, R. Brito, and K.J. McClellan}

\tab Current nuclear reactors face stiff competition in energy markets from far less capital intensive power plants, such as coal or natural gas. One way to maximize the return on investment for current nuclear plants is to facilitate a power uprate for the reactor, or reduce refueling outages. The article by Brown \textit{et al.} explores how using uranium nitride (UN) fuel could maximize the output of PWRs as well as UN's compatability with contemporary reactor designs. In their analysis, the authors performed several simulations, first using TRITON and Serpent to generate applicable constants, and then using PARCS to use those parameters to model reactor cycles using UN in comparison to UO$_2$. Notably, their study showed that the UN fuel has significantly longer burn-up times than UO$_2$. \\
\tab I think there were a couple of places the paper could have improved upon its descriptions. First, I was interested to note that the simulation did not include control rods, rather considering only integral fuel burnable absorbers (IFBAs). This strikes me as a curious choice since using control rods is standard practice in PWRs, and the article seems geared towards developing UN fuel as a replacement for UO$_2$ in current reactors. Secondly I would have like to see a comparison of how the core cycle extension production values compare with projections of processing costs for UN fuel. 
\end{document}


Performed analytic assessment of uranium nitride (UN) fuels
Analysis/few-group constant generation using TRITON and Serpent
Full core modeling performed using Purdue Advanced Reactor Core Simulator (PARCS)
Beginning of Life benchmark shows similarity in calculation results of codes
UN produces longer cycle lengths than UO2
UN requires more expensive processing for fabrication
PARCS core simulator used two-group parameters generated by TRITON
Good visuals showing core shuffle
Interesting simulation choice not to include control rods or WABAs in calculation
Use IFBA poison rods, not control rods
UN samples all have less negative reactivity coefficients (can't just drop them into a UO2 PWR w/ same safety evaluations)

