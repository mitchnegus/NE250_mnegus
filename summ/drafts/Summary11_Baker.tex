\documentclass{report}
% PACKAGES %
\usepackage[english]{} % Sets the language
\usepackage[margin=2cm]{geometry} % Sets the margin size
\usepackage{fancyhdr} % Allows creation of headers
\usepackage{xcolor} % Allows the use of color in text
\usepackage{float} % Allows figures and tables to be floats
\usepackage{appendix}
\usepackage{amsmath} % Enhanced math package prepared by the American Mathematical Society
	\DeclareMathOperator{\sech}{sech} % Include sech
\usepackage{amssymb} % AMS symbols package
\usepackage{mathrsfs}% More math symbols
\usepackage{bm} % Allows you to use \bm{} to make any symbol bold
\usepackage{bbold} % Allows more bold characters
\usepackage{verbatim} % Allows you to include code snippets
\usepackage{setspace} % Allows you to change the spacing between lines at different points in the document
\usepackage{parskip} % Allows you alter the spacing between paragraphs
\usepackage{multicol} % Allows text division into multiple columns
\usepackage{units} % Allows fractions to be expressed diagonally instead of vertically
\usepackage{booktabs,multirow,multirow} % Gives extra table functionality
\usepackage{hyperref} % Allows hyperlinks in the document
\usepackage{rotating} % Allows tables to be rotated
\usepackage{graphicx} % Enhanced package for including graphics/figures
	% Set path to figure image files
	%\graphicspath{ } }
\usepackage{listings} % for including text files
	\lstset{basicstyle=\ttfamily\scriptsize,
        		  keywordstyle=\color{blue}\ttfamily,
        	  	  stringstyle=\color{red}\ttfamily,
          	  commentstyle=\color{gray}\ttfamily,
          	 }		
\newcommand{\tab}{\-\hspace{1cm}}

% Create a header w/ Name & Date
\pagestyle{fancy}
\rhead{\textbf{Mitch Negus} \; 11/9/2017}

\begin{document}
\thispagestyle{empty}

{\bf {\large {NE250 Summary {11} \hfill Mitch Negus\\
		\hspace*{\fill} 11/9/2017\\ }}}
\section*{\textsl{An S_n Algorithm for the Massively Parallel CM-200 Computer} \\ \normalsize R. Baker and K. Koch}

\tab ...  \\
\tab ...


\end{document}

-2048 processors is small today (how did that compare at the time)
-CIMD architecture requires all operations to be the same per step
	- why is this (assuming circuit logic makes this a simple/fast design?)	
-2D diagonal line sweep, O(N/2); 3D diagonal plane sweep O(N^2/3)
-3D plane projected onto YZ plane for parallelization
	- it is one-to-one (not necessarily onto)
-for inactive X, the mapping from YZ plane to XYZ space gives an invalid X
	- fake cell layer
	- can still execute procedure and solve in required geometry
	- more computationally efficient than WHERE mask
-parallel computation efficiency (PCE) of only ~1/3
	- used a pipelining approach to parallelize more efficiently
	- 2 methods
		- ``Successive angles, successive quadrants''
			- (-) angle m, (+) angle m, (-) angle (m+1), (+) angle (m+1)
		- ``Simultaneous angle, successive quadrants''
-tested algorithm against experimental spherical reflected core
	-uranium core
	-beryllium reflector
-homogenized boundary cells to reconcile spherical geometry w/ cubic mesh
-used diamond difference w/ and w/o set-to-zero (STZ) fix up
	- fix-up is significantly more expensive computationally
	- results in load imbalances and poor parallelization
-128x128x128 mesh shows good scalability 
	- speedup ratio of 1.88 for 512-->1024 proc's; theoretical limit is 2
-is the comparison of a single CRAY processor to CM-200 processor reasonable/valid?
	