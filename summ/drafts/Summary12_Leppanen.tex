\documentclass{report}
% PACKAGES %
\usepackage[english]{} % Sets the language
\usepackage[margin=2cm]{geometry} % Sets the margin size
\usepackage{fancyhdr} % Allows creation of headers
\usepackage{xcolor} % Allows the use of color in text
\usepackage{float} % Allows figures and tables to be floats
\usepackage{appendix}
\usepackage{amsmath} % Enhanced math package prepared by the American Mathematical Society
	\DeclareMathOperator{\sech}{sech} % Include sech
\usepackage{amssymb} % AMS symbols package
\usepackage{mathrsfs}% More math symbols
\usepackage{bm} % Allows you to use \bm{} to make any symbol bold
\usepackage{bbold} % Allows more bold characters
\usepackage{verbatim} % Allows you to include code snippets
\usepackage{setspace} % Allows you to change the spacing between lines at different points in the document
\usepackage{parskip} % Allows you alter the spacing between paragraphs
\usepackage{multicol} % Allows text division into multiple columns
\usepackage{units} % Allows fractions to be expressed diagonally instead of vertically
\usepackage{booktabs,multirow,multirow} % Gives extra table functionality
\usepackage{hyperref} % Allows hyperlinks in the document
\usepackage{rotating} % Allows tables to be rotated
\usepackage{graphicx} % Enhanced package for including graphics/figures
	% Set path to figure image files
	%\graphicspath{ } }
\usepackage{listings} % for including text files
	\lstset{basicstyle=\ttfamily\scriptsize,
        		  keywordstyle=\color{blue}\ttfamily,
        	  	  stringstyle=\color{red}\ttfamily,
          	  commentstyle=\color{gray}\ttfamily,
          	 }		
\newcommand{\tab}{\-\hspace{1cm}}

% Create a header w/ Name & Date
\pagestyle{fancy}
\rhead{\textbf{Mitch Negus} \; 11/17/2017}

\begin{document}
\thispagestyle{empty}

{\bf {\large {NE250 Summary 12} \hfill Mitch Negus\\
		\hspace*{\fill} 11/17/2017\\ }}
\section*{\textsl{Performance of Woodcock delta-tracking in lattice physics applications using the Serpent Monte Carlo reactor physics burnup calculation code} \\ \normalsize J. Lepp\"{a}nen}

\tab ...


\end{document}

-Woodcock Delta tracking
	-assign majorant cross section
	-use XS_maj w/ a rejection sampling alogorithm
	-less comp. intensive than recaculating neutron path length in each material
	-algorithm becomes materially invariant; acceptance probability depends on material
-some limitations to delta-tracking
	-reaction rate integrals difficult to calculate
	-typical track-length estimator invalid
	-collision flux estimator inefficient
	-heavy absorbers (high difference in XS locally)
	-many more virtual collisions take place than necessary (most are wasted)
-implemented a cutoff alue to determine "how much" delta tracking to use
	-jumps/discontinuities in plots suggest there are good places to put cutoff value to optimize
	-depends on geometry
-delta tracking w/ optimal cutoff produces speed-up of 2-13 times
	-~2 for BWR/PWR
	-~13 for HTGR

-What is the savings of this algorithm (order?)
	