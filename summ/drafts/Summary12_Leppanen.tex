\documentclass{report}
% PACKAGES %
\usepackage[english]{} % Sets the language
\usepackage[margin=2cm]{geometry} % Sets the margin size
\usepackage{fancyhdr} % Allows creation of headers
\usepackage{xcolor} % Allows the use of color in text
\usepackage{float} % Allows figures and tables to be floats
\usepackage{appendix}
\usepackage{amsmath} % Enhanced math package prepared by the American Mathematical Society
	\DeclareMathOperator{\sech}{sech} % Include sech
\usepackage{amssymb} % AMS symbols package
\usepackage{mathrsfs}% More math symbols
\usepackage{bm} % Allows you to use \bm{} to make any symbol bold
\usepackage{bbold} % Allows more bold characters
\usepackage{verbatim} % Allows you to include code snippets
\usepackage{setspace} % Allows you to change the spacing between lines at different points in the document
\usepackage{parskip} % Allows you alter the spacing between paragraphs
\usepackage{multicol} % Allows text division into multiple columns
\usepackage{units} % Allows fractions to be expressed diagonally instead of vertically
\usepackage{booktabs,multirow,multirow} % Gives extra table functionality
\usepackage{hyperref} % Allows hyperlinks in the document
\usepackage{rotating} % Allows tables to be rotated
\usepackage{graphicx} % Enhanced package for including graphics/figures
	% Set path to figure image files
	%\graphicspath{ } }
\usepackage{listings} % for including text files
	\lstset{basicstyle=\ttfamily\scriptsize,
        		  keywordstyle=\color{blue}\ttfamily,
        	  	  stringstyle=\color{red}\ttfamily,
          	  commentstyle=\color{gray}\ttfamily,
          	 }		
\newcommand{\tab}{\-\hspace{1cm}}

% Create a header w/ Name & Date
\pagestyle{fancy}
\rhead{\textbf{Mitch Negus} \; 11/17/2017}

\begin{document}
\thispagestyle{empty}

{\bf {\large {NE250 Summary 12} \hfill Mitch Negus\\
		\hspace*{\fill} 11/17/2017\\ }}
\section*{\textsl{Performance of Woodcock delta-tracking in lattice physics applications using the Serpent Monte Carlo reactor physics burnup calculation code} \\ \normalsize J. Lepp\"{a}nen}

\tab This week's article discusses the benefits of using Woodcock delta tracking in the open source Monte Carlo code, Serpent. Delta tracking solves some of the problems that Monte Carlo algorithms experience in regards to having to calculate new track lengths each time a particle transitions from one material region in a geometry to another. To avoid these expensive calculations, delta tracking uses a majorant cross section (the largest cross section in the problem) to calculate path lengths. Collisions are then weighted  based on the true cross section of the material (sometimes being rejected as virtual collisions, otherwise counted as a true collision). When materials have generally similar cross section, this procedure saves much of the time spent calculating boundary transtions. Evidence of this enhancement is given in the article, where Lepp\"{a}nen shows an increase in speed for almost all cases tested with what is determined to be the optimal amount of delta tracking for the problem. Some geometries, like BWRs and PWRs see an increase in speed of only about 2 times, while others like HTGRs see a speed increase by a factor of about 13.\\
\tab I thought that on the whole, the article was well presented and fairly readable. It had a good distribution of background information, motivation, a brief discussion of limitations and results, but was also concise enough that it was not unwieldy. I think it might have been interesting if the author had described what the order of calculations might be for a sample problem ($\mathcal{O}(n^2)$ for traditional MC, but $\mathcal{O}(n)$ for delta tracking). That being said, this calculation would be geometry dependent, so such a calculation might be prohibitively difficult for anything more than one or two simple geometries.


\end{document}

-Woodcock Delta tracking
	-assign majorant cross section
	-use XS_maj w/ a rejection sampling alogorithm
	-less comp. intensive than recaculating neutron path length in each material
	-algorithm becomes materially invariant; acceptance probability depends on material
-some limitations to delta-tracking
	-reaction rate integrals difficult to calculate
	-typical track-length estimator invalid
	-collision flux estimator inefficient
	-heavy absorbers (high difference in XS locally)
	-many more virtual collisions take place than necessary (most are wasted)
-implemented a cutoff alue to determine "how much" delta tracking to use
	-jumps/discontinuities in plots suggest there are good places to put cutoff value to optimize
	-depends on geometry
-delta tracking w/ optimal cutoff produces speed-up of 2-13 times
	-~2 for BWR/PWR
	-~13 for HTGR

-What is the savings of this algorithm (order?)
	